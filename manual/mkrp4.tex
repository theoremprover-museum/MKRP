\chapter{The MKRP-Editor}
\label{TheMKRPEditor}


{\sc Markgraf Karl} has an own syntax-directed editor for the 
sorted logic formulae, written in PLL, but it is not advisable to use it 
in its full extent. At the end of the chapter we give hints  how to handle
 the formula input on different machines. On Symbolics computers the Zmacs 
editor is integrated into the system. 

In the {\sc Mkrp}-editor the formulae are kept in two 
different areas: if a formula was accepted by the compiler, it is included 
into the active area. In this case symbol table entries and prefix form 
exist for the formula and are accessible.
Other formulae (e.g.\ such with syntax errors) are stored in the passive 
area. Thus even erroneous input is not lost and the passive area can be 
used as a scratch pad. 

When terminated the editor returns the list of the formulae in the active 
area for further processing by the theorem prover. The passive 
formulae are not considered. 

Below the editor commands are listed. Several commands can be
concatenated by the separator {\tt \_}. Every command must begin with
an atom, the command name. In many cases the command name can be
omitted for the INSERT command.

\newcommand\edc[2]{\pagebreak[3]{\bf #1}\\[1mm] #2}

\index{I[NSERT]}\index{editor command I[NSERT]}\edc{I[NSERT] $<$FORMULA$>$ \\
$<$FORMULA$>$}%                
{If formula $<$FORMULA$>$ is syntactically and semantically correct, it
 will be 
inserted as last one into the active area, else as first one into the
 passive area.}

\index{D[ELETE]}\index{editor command D[ELETE]}\edc{D[ELETE] [$<$FROM$>$] [--] [$<$TO$>$]\\
D[ELETE] [$<$FROM$>$] [/] [$<$TO$>$]}%                
{If $<$TO$>$ is greater than or equal to the number of the last formula in 
the active area, the 
formulae from $<$FROM$>$ to $<$TO$>$ will be deleted. 

\begin{tabular}{lll}
Examples:&	D--     &	deletes all formulae, \\
&	D 2&	deletes formula 2,\\
&	D 3--&	deletes all formulae from the third one, \\
&	D&	deletes the last formula in the active area if it exists, \\
&& else the first one in the passive area. 
\end{tabular}}

\index{+[SHIFT]}\index{editor command +[SHIFT]}\edc{+[SHIFT] [$<$NUMBER$>$]\\
++[SHIFT]}%                
{+SHIFT shifts the first $<$NUMBER$>$ formulae of the passive area into the 
active  area, if they are syntactically and semantically correct. Default
 of $<$NUMBER$>$ is 1. 
++SHIFT shifts all formulae. 
\begin{tabular}{lccc}
Example: &	* 1 *       &&                   * 1 *\\
&	   2        &    + 2   &         * 2 *\\
&	   3   &  $\longrightarrow$   &   * 3 *\\
&	   4      &&                         4
\end{tabular}}

\index{--[SHIFT]}\index{editor command --[SHIFT]}\edc{--[SHIFT] 
[$<$NUMBER$>$]\\
-- --[SHIFT]}%                
{--SHIFT shifts the last $<$NUMBER$>$ formulae of the active area into the 
passive area. Default of $<$NUMBER$>$ is 1. -- --SHIFT shifts all formulae. 

\begin{tabular}{lccc}
Example:	&* 1 *      &     --      &        * 1 *\\
	&* 2 * &   $\longrightarrow$  &      2
\end{tabular}}

\index{E[DIT]}\index{editor command E[DIT]}\edc{E[DIT] [$<$NR$>$]}%
{The Lisp-Editor is called for the formula $<$NR$>$.
If the active area is not empty, the default of $<$NR$>$ is the number of 
the last formula in the active area, else 1.}

\index{C[HANGE]}\index{editor command C[HANGE]}\edc{C[HANGE] [$<$NR$>$]}%
{The formula $<$NR$>$ is printed on the terminal. The user should enter a new 
formula to replace the printed one. 
If the active area is not empty, the default of $<$NR$>$ is the number of the
 last 
formula in the active area, else 1. }

\index{R[EAD]}\index{editor command R[EAD]}\edc{R[EAD] $<$FILE$>$}%
{$<$FILE$>$ must be created by the WRITE-command or can alternatively be in
 the format acceptable to the GET-command. If the editor is in the 
initial state, the formulae are inserted into the same areas containing 
them at write time or are tried to be shifted into the active area if they
 are acceptable to the GET-command. Otherwise all formulae are inserted into
 the passive 
area. }

\index{W[RITE]}\index{editor command W[RITE]}\edc{W[RITE] $<$FILE$>$}%
{The contents of the editor will be saved on file $<$FILE$>$, so that it can
 be 
restored with the read-command. }

\index{W[RITE]G[ET]}\index{editor command W[RITE]G[ET]}\edc{W[RITE]G[ET] $<$FILE$>$}%
{The contents of the active area is written on file $<$FILE$>$ in a format
 acceptable to the GET-command.}

\index{G[ET]}\index{editor command G[ET]}\edc{G[ET] $<$FILE$>$}%
{$<$FILE$>$ contains a sequence of formulae in the following format:
\begin{center}
($<$FORMULA$_1$$>$)\\
($<$FORMULA$_2$$>$)\\
$\vdots$\\
($<$FORMULA$_n$$>$)
\end{center}
The formulae are inserted into the passive area. If the system is
installed with communication to an editor (e.g.\ Zmacs on Symbolics)
they are then all shifted into the
active area exactly as if SHIFT++ is executed. }

\index{EXEC[UTE]}\index{editor command EXEC[UTE]}\edc{EXEC[UTE] $<$FILE$>$}%
{The file $<$FILE$>$ contains a sequence of editor commands in the following 
format:
\begin{center}
$<$COMMAND$_1$$>$ {\tt \_}\\
$<$COMMAND$_2$$>$ {\tt \_}\\
$\vdots$\\
$<$COMMAND$_n$$>$ {\tt \_}\\
OK
\end{center}
The commands will be executed. For further dialogue the terminal is used.
Take care of the right number of closing parentheses. }

\index{SW[ITCH]}\index{editor command SW[ITCH]}\edc{SW[ITCH] [$<$NR$_1$$>$] [$<$NR$_2$$>$]}%
{If formula $<$NR$_1$$>$ and formula $<$NR$_2$$>$ are in the passive area,
they will be exchanged. 
If the user only gives one number, this and the first formula are taken,
switch alone exchanges the first and the last formula of the passive area. }

\index{U[NDO]}\index{editor command U[NDO]}\edc{U[NDO] [$<$NUMBER$>$]\\
U[NDO] ON\\
U[NDO] OFF}%                
{UNDO ON and UNDO OFF are switching undo-mode ON or OFF, respectively. 
UNDO undoes the last $<$NUMBER$>$ of destructive commands,  as INSERT, 
DELETE, EDIT, CHANGE, +SHIFT, --SHIFT, READ, and GET.
Giving no $<$NUMBER$>$ means one command. }

\index{REP[LACE]}\index{editor command REP[LACE]}\edc{REP[LACE] $<$OLD$>$ $<$NEW$>$\\
X $<$OLD$>$ $<$NEW$>$}%                
{REPLACE replaces the symbol $<$OLD$>$ by the symbol $<$NEW$>$ in the active
and 
passive area as well as in the symbol-table. REPLACE can not be undone, nor 
can be any of the commands executed prior to a replace. }

\index{PP[RINT]}\index{editor command PP[RINT]}\edc{PP[RINT] $<$FILE$>$}%
{PPRINT writes the symbol-table and all formulae in a readable form on the 
given file. }

\index{S[HOW]}\index{editor command S[HOW]}\edc{S[HOW] $<$X$_1$$>$...$<$X$_n$$>$\\
S[HOW] --}%                
{Each $<$X$_i$$>$ must be a symbol-name or a list of kinds. 
There are the kinds S[ORT], F[UNCTION], P[REDICATE], and C[ONSTANT]. All 
symbols that are so specified will be shown on the terminal. 

\begin{tabular}{lll}
Example:&	S (P) H G & will show all predicates and the symbols H and G,\\
	& S--      &  all symbols. 
\end{tabular}}

\index{SS[HOW]}\index{editor command SS[HOW]}\edc{SS[HOW] 
$<$X$_1$$>$...$<$X$_n$$>$\\
SS[HOW] --}%                
{SSHOW is a more beautyful version of show.
Each $<$X$_i$$>$ must be a symbol-name or a list of kinds. 
There are the kinds S[ORT], F[UNCTION], P[REDICATE], and C[ONSTANT]. All 
symbols that are so specified will be shown on the terminal.

\begin{tabular}{lll}
Example:&	SS (P) G & will show all predicates and symbol G,\\
	& SS--      &  all symbols. 
\end{tabular}}

\index{PRE[FIX]}\index{editor command PRE[FIX]}\edc{PRE[FIX] [$<$FROM$>$] [--] [$<$TO$>$]\\
PRE[FIX] [$<$FROM$>$] [/] [$<$TO$>$]\\
F [$<$FROM$>$] [--] [$<$TO$>$]\\
F [$<$FROM$>$] [/] [$<$TO$>$]}%                
{PREFIX or F writes the compiled (prefix) form of the formulae $<$FROM$>$  to 
$<$TO$>$ on the screen. 
\begin{tabular}{lll}
Examples:&	PRE-- or  PRE/  & all formulae of the active area,\\
&	PRE 3        &       formula 3, if active, \\
&	PRE --5        &     the formulae 1 to 5, if they are in the active 
area.
\end{tabular}}



\index{IN[FIX]}\index{editor command IN[FIX]}\edc{IN[FIX] [$<$FROM$>$] [--] [$<$TO$>$]\\
IN[FIX]  [$<$FROM$>$] [/] [$<$TO$>$]\\
L[IST]   [$<$FROM$>$] [--] [$<$TO$>$]\\
L[IST]   [$<$FROM$>$] [/] [$<$TO$>$]}%                
{ INFIX or LIST writes the infixform (input-language) of the formulae 
$<$FROM$>$ to $<$TO$>$ on the screen.

\begin{tabular}{lll}
Examples: &	IN-- or IN/  & all formulae,\\
&	IN7	&   the seventh formula, \\
&	IN --5	&   the formulae 1 to 5\\
&	IN	 &  the last formula of active area, or if it doesn't 
exist \\
&& the first one of the passive area. 
\end{tabular}}

\index{ST[ATE]}\index{editor command ST[ATE]}\edc{ST[ATE]}%
{STATE shows the user, how the formulae are distributed on the areas. 

\begin{tabular}{llll}
Example:	& AREA 1: & 3   formulae   &   (active area) \\
&	AREA 0: & 1   formula     &   (passive area)
\end{tabular}}

\index{OK}\index{editor command OK}\edc{OK}%
{OK terminates the editor and returns to the calling module. 
OK used on a file for the execute command terminates the execute 
command. }

\index{!}\index{editor command !}\edc{!}%
{! terminates the editor and returns to the {\sc Mkrp}-operatingsystem loop. 
! in the input of a command cancels it.} 

\index{LISP}\index{editor command LISP}\edc{LISP}%                
{LISP  calls the LISP system.
You can return to the editor by OK.}

\index{V}\index{editor command V}\edc{V ON\\
V [OFF]}%                
{V ON and V T switch on the possibility to manually control the scrolling of the screen (``more mode''). 
V OFF and V switch it off. This doesn't work for all machines.} 

\index{H[ELP]}\index{editor command H[ELP]}\edc{H[ELP] [$<$COMMAND$>$]}%     
{HELP  prints a list of all possible commands on the screen. 
HELP $<$COMMAND$>$ explaines the command $<$COMMAND$>$.}

\index{HH[ELP]}\index{editor command HH[ELP]}\edc{HH[ELP] $<$FILE$>$}%        
{HHELP prints explanations for all possible commands on file
$<$FILE$>$ corresponding to this chapter of the manual.}




In the version running on Symbolics machines, the editor Zmacs should be used to 
edit the formulae. It is called from the editor subsystem of {\sc Mkrp} by the 
command I[NSERT] or G[ET] without a parameter.

In both cases every formula has to be enclosed in parentheses as specified for the GET-command. It is then 
actually inserted into the active area of {\sc Mkrp} by marking it for example
with the mouse and typing the key $<$END$>$. If more 
than one formula is marked, all of them are entered in order.
