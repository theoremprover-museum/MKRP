\chapter{Setting the MKRP Parameters}
\label{SettingtheMKRPParameters}


First it should be stated that for most problems the coding and not
the option setting is the basis of automatically finding a proof, just
as it is in programming: Giving an optimize option to the compiler
very rarely leads to a drastically faster algorithm. The effect is
even more remarkable when using an automatic theorem prover (see also
\cite{KePr92}).

The OPTIONS command enters the options module. It offers the
possibility to adjust various parameters to govern the overall search
behaviour and influence trace and protocol. The options are divided
into various areas according to their tasks.


\section{Explanation of the Commands}

In the options module, the following commands are available:

H[ELP]  P[RINT]  PP[RINT]  R[EAD]  W[RITE]  OK          V

On window capable systems these commands are available via menus.

\begin{tabbing}
PP[RINT]  $<$File$>$ \=\kill
H[ELP]\> prints a list of all available commands \\
H[ELP]  $<$Com$>$ 	\> explains the command $<$Com$>$.\\
\\
P[RINT] $<$Area$>$ 	\> prints the options of $<$Area$>$ and their current values.
\\
PP[RINT]  $<$File$>$\> prints all areas , their options and default values 
together\\
\> with a detailed explanation on $<$File$>$.\\ 
\>$<$File$>$ is handled like the file arguments of the operating system commands.\\
\>It is written into the directory specified by DD and DE.\\
\\
R[EAD] $<$File$>$   \> 	 reads the option-values from $<$File$>$. This file 
must have been \\
\> created  by a W[RITE] $<$File$>$ command.\\
\\
W[RITE] $<$File$>$  \> writes all options and their current values on $<$File$>$.\\
\\
V  T/Y/YES/OK   \> 	 turns the manual terminal control on. \\
V  NIL/N/NO     \>	 turns it off again. This doesn't work for all
machines.\\
\\
$<$Area$>$       \>       prints all the options of the area $<$Area$>$\\
\\
$<$Opt$>$ $<$Val$>$   \>    sets the option $<$Opt$>$ to the new value $<$Val$>$
\end{tabbing}

\section{Explanation of the Options}

The purpose of the options is to adjust the general search behaviour of the 
theorem prover to the characteristics of the given problem. According  to 
their tasks the options are grouped in the following categories: 

 \quad     TWO   RED.I   RED.D   STR   SORT   ER  GEN   TR   PR

All options have default values. 

To influence the construction of the initial connection graph (CONSTRUCT) 
the user can change the values of the options of TWO, RED.I, ER, SORT, and the option 
SPLITTING of the area GEN.

The search behaviour during the deduction process (REFUTE) can be 
governed by the options of the areas RED.D, ER, SORT, and STR. 

Tracing of the deduction process can be influenced by options of the area 
TR. The information on the protocol can be selected by options of the area 
PR (options of the areas TR and PR are discussed in detail in chapter \ref{TheOutputFacilities}).

The options of the area ER enable the user to control equality reasoning
for construction and refutation.

All other options which do not fit in any of these areas are put into the 
area GEN.

To solve difficult problems it seems to be very useful to change the values 
of the options:

\begin{tabbing}
STR\_TERM.BREADTH.FIRST\quad \=\kill
        STR\_R.SELECTION \> (see page \pageref{strrselection})\\
	ER\_PARAMODULATION \> (see page \pageref{erparamodulation})\\
        ER\_WEIGHT.POLYNOMIALS \> (see page \pageref{erweightpolynomials})\\
        ER\_P.SELECTION \> (see page \pageref{erpselection})\\
	ER\_NARROW.DEPTH  \> (see page \pageref{ernarrowdepth})\\
	ER\_COMPLETION  \> (see page \pageref{ercompletion})\\
	TERM\_UNITS  \> (see page \pageref{termunits})\\
 	TERM\_ITERATIONS  \> (see page \pageref{termiterations})\\
        GEN\_MANUAL.CONTROL \> (see page \pageref{genmanualcontrol})
\end{tabbing}

Advantages and disadvantages of the various option settings are discussed 
below. Reduction operations should be switched off for efficiency reasons 
if it is known that there is no possibility for applying a reduction 
operation (especially link reductions).

In the following description some effects of options of the areas TWO, 
RED.I, and RED.D are demonstrated by examples of simple refutation graphs. 
The default values of options are marked with *.


\subsection{Options of the Category TWO}

This category consists of options handling two-literal clauses during the 
construction of the initial connection graph in a special way. The special 
handling results in the insertion of additional links representing a 
sequence of two or more deduction steps. 

\subsubsection{TWO\_RULES}
\index{{\sc two\_rules}}
\index{rules} \index{option rules} \index{two option rules}

The value of this option controls the special handling of two-literal
clauses. 

\begin{samepage}
\Ex

\begin{enumerate}
\item
Resolving link $L_1$ in figure \ref{TwoLiteralClause}a and
inheriting link $L_2$
creates the situation of figure \ref{TwoLiteralClause}b. Another resolution 
leads to the clause of figure \ref{TwoLiteralClause}c.
                                     
This deduction process can be abbreviated by inserting a link between $C_2$ 
and $C_3$ as done in figure \ref{TwoLiteralClause}d. Resolving on this new link has the same effect as resolving on $L_3$, 
but needs only one resolution step.   

\begin{figure}[ht]
\caption{Two-Literal Clauses}
\label{TwoLiteralClause}
\hfill
\parbox[t]{7cm}{
\begin{picture}(64,10)
\put(0,7){a)}
% Klauselnamen
\put(6,5){$C_2$}
\put(30,5){$C_1$}
\put(54,5){$C_3$}
% Linknamen
\put(18,3){$L_1$}
\put(42,3){$L_2$}
% Links
\put(16,2){\line(1,0){8}}
\put(40,2){\line(1,0){8}}
% Letzte Zeile
\put(0,0){\framebox(8,4){$T$}}
\put(8,0){\framebox(8,4){$Pab$}}
\put(24,0){\framebox(9,4){$\neg Pxy$}}
\put(33,0){\framebox(7,4){$Ryx$}}
\put(48,0){\framebox(9,4){$\neg Rba$}}
\put(57,0){\framebox(7,4){$S$}}
\end{picture}\\[3mm]
\begin{picture}(64,10)
\put(0,7){b)}
% Klauselnamen
\put(6,5){$C_4$}
\put(54,5){$C_3$}
% Linknamen
\put(30,3){$L_3$}
% Links
\put(16,2){\line(1,0){32}}
% Letzte Zeile
\put(0,0){\framebox(8,4){$T$}}
\put(8,0){\framebox(8,4){$Rba$}}
\put(48,0){\framebox(9,4){$\neg Rba$}}
\put(57,0){\framebox(7,4){$S$}}
\end{picture}}
\hfill
\parbox[t]{7cm}{
\begin{picture}(16,10)
\put(0,7){c)}
% Klauselnamen
\put(6,5){$C_5$}
% Letzte Zeile
\put(0,0){\framebox(8,4){$T$}}
\put(8,0){\framebox(8,4){$S$}}
\end{picture}\\[3mm]
\begin{picture}(64,10)
\put(0,7){d)}
% Klauselnamen
\put(6,5){$C_2$}
\put(54,5){$C_3$}
% Linknamen
\put(30,3){$L_4$}
% Links
\put(16,2){\line(1,0){32}}
% Letzte Zeile
\put(0,0){\framebox(8,4){$T$}}
\put(8,0){\framebox(8,4){$Pab$}}
\put(48,0){\framebox(9,4){$\neg Rba$}}
\put(57,0){\framebox(7,4){$S$}}
\end{picture}}
\hfill
\end{figure}

                                        
\item By this method we can save more than one resolution step as 
shown in figure \ref{MoreTwoLiteralClause}
    ($f$ is a binary function symbol).  The links with strokes are 
replaced by the one without.
By this way, chains of two-literal clauses of any length can be simulated 
by one link. 

\begin{figure}[ht]
\caption{More Two-Literal Clauses}
\label{MoreTwoLiteralClause}
\begin{center}
\begin{picture}(46,39)
% Linkdurchstreichungen
\put(32,12){\line(1,-1){4}}
\put(10,28){\line(1,-1){4}}
\put(24,26){\line(1,0){4}}
% Links
\put(6,16){\line(1,-1){14}}
\put(28,4){\line(1,1){12}}
\put(6,20){\line(1,1){12}}
\put(26,32){\line(0,-1){12}}
% Klauseln Oben
\put(12,32){\framebox(10,4){$\neg Pxy$}}
\put(22,32){\framebox(8,4){$Ryx$}}
% Mitte 
\put(0,16){\framebox(15,4){$Pfafbcd$}}
\put(18,16){\framebox(20,4){$\neg Rufxfyz$}}
\put(38,16){\framebox(15,4){$Puffxyz$}}
% Letzte Zeile
\put(20,0){\framebox(16,4){$\neg Pdffabc$}}
\end{picture}
\end{center}
\end{figure}

\item The special handling of two-literal clauses can also connect a clause 
with        itself as in the situation in figure \ref{SelfResolvent}. 

In the original graph without the circle link it is possible to deduce 
the empty 
clause. In the graph resulting from the special handling of two-literal
clauses this is only possible when resolving on a circle link. 

\begin{figure}[ht]
\caption{Self Resolvent}
\label{SelfResolvent}
\begin{center}
\begin{picture}(16,24)
% Linkdurchstreichungen
\put(3,13){\line(1,0){10}}
% Links
\put(8,8){\line(-1,3){4}}
\put(8,8){\line(1,3){4}}
\put(8,4){\oval(6,3)[b]}
% Klauseln Oben
\put(0,20){\framebox(8,4){$Pxy$}}
\put(8,20){\framebox(8,4){$Pyx$}}
% Letzte Zeile
\put(3,4){\framebox(10,4){$\neg Paa$}}
\end{picture}
\end{center}
\end{figure}

\end{enumerate}
\end{samepage}

\PO

\oline{T/Y/YES}{Two-literal clauses are treated as described above.}
\oline{PARTIAL}{Link insertion only if the two clauses connected by the new 
		link are different.}
\oline{$^*$NIL/N/NO}{The two-literal clause algorithm is switched off.}

\paragraph{Effects of the Special Handling of Two-Literal Clauses}

The advantages of the special handling are at the one hand that the proof 
procedure needs less deduction steps to deduce the empty clause, on the 
other hand all intermediate clauses are not inserted 
in the graph.
Switching on the two-literal rule algorithm may  cause the system to 
consume more time. The completion of the graph by this way may cause an
increasing size of the graph. 

The computations enforced by the of two-literal clauses are
executed during the construction of the initial connection graph
(CONSTRUCT) but take effect to the deduction steps during REFUTE.

\paragraph{Remark}
This option is incompatible with the equality predicate and its setting is 
ignored if equations are present.

\subsubsection{TWO\_SUPPRESS.NO.RULES}
\index{{\sc two\_suppress.no.rules}}\index{suppress no rules} 
\index{option suppress no rules} \index{two option suppress no rules}

This option controls what to do if there 
are chains of two-literal clauses which are longer  than the maximal 
length. 

\PO
\oline{T/Y/YES}{The clauses are explicitly created and inserted into the 
connection graph.}
\oline{$^*$NIL/N/NO}{The creation of these clauses is suppressed and 
therefore the potential links are ignored.}

\paragraph{Remark}
With value NIL for this option the proof procedure becomes incomplete. 

 
\subsubsection{TWO\_RULES.MAXLEVEL}
\index{{\sc two\_rules.maxlevel}}
\index{rules maxlevel}\index{option rules maxlevel} 
\index{two option rules maxlevel}

The value of this option is the maximal length of two-literal clause
chains which are substituted by one link. 

\PO 
\olineeins{Natural Numbers}
\oline{Default Value}{1}


\subsection{Options of the Category RED.I}

This area contains options which influence the construction of the initial 
connection graph

\subsubsection{RED.I\_CLAUSE.MULTIPLE.LITERALS}
\index{{\sc red.i\_clause.multiple.literals}}
\index{clause multiple literals} \index{option clause multiple literals} \index{initial reduction option clause multiple literals}

A clause containing identical literals can be simplified by deletion 
of the multiple literals. Two literals are identical if they are connected 
by an S2-link with empty substitution. With two-literal rules switched off, that is just the case when they have 
\begin{itemize}
\item the same sign,
\item the same predicate symbol, and
\item equal term lists.
\end{itemize}
(equal means here equal under a certain theory)

Figure \ref{MultipleLiterals} shows examples,                   
where $R$ is symmetric and $f$ is associative.

\begin{figure}[ht]
\caption{Multiple Literals}
\label{MultipleLiterals}
\begin{center}
\begin{picture}(48,4)
\put(0,0){\framebox(12,4){$Pa$}}
\put(12,0){\framebox(12,4){$Pa$}}
\put(28,2){$\rightarrow$}
\put(36,0){\framebox(12,4){$Pa$}}
\end{picture}
\qquad\qquad
\begin{picture}(64,4)
\put(0,0){\framebox(16,4){$Rafxfyz$}}
\put(16,0){\framebox(16,4){$Rffxyza$}}
\put(38,2){$\rightarrow$}
\put(48,0){\framebox(16,4){$Rafxfyz$}}
\end{picture}
\end{center}
\end{figure}



\PO
\oline{$^*$T/Y/YES}{Multiple literals are removed.}
\oline{NIL/N/NO}{Clauses with multiple literals remain unchanged.}

\subsubsection{RED.I\_CLAUSE.PURITY}
\index{{\sc red.i\_clause.purity}}
\index{clause purity} \index{option clause purity} \index{initial reduction option clause purity}

This option offers the possibility to delete all clauses of the connection 
graph which cannot contribute to the deduction of the empty clause, because 
they are ``pure''. A clause is pure, if and only if it contains a literal which 
is not connectecd by an R1-link. This means no deduction step is 
possible to reduce a pure clause to the empty clause, i.e.\ pure clauses 
cannot support the deduction of the empty clause.




\Ex

The clause $C_1$ of figure \ref{PureLiterals} is pure because the literal 
$S$ is not connected by an R-link and is therefore eliminated. All links 
connecting $C_1$ to other clauses 
are also deleted.             
  
            
\begin{figure}[ht]
\caption{Pure Literals}
\label{PureLiterals}
\begin{center}
\begin{picture}(56,22)
\put(0,18){$C_1$}
\put(4,16){\framebox(6,4){$S$}}
\put(10,16){\framebox(6,4){$R$}}
\put(13,16){\line(0,-1){2}}
\put(13,16){\line(-1,-1){2}}
\put(13,16){\line(1,-1){2}}
\put(16,16){\framebox(6,4){$O$}}
\put(19,20){\line(0,1){2}}
\put(22,16){\framebox(6,4){$Qy$}}
\put(25,20){\line(0,1){2}}
\put(28,16){\framebox(6,4){$Px$}}
\put(31,16){\line(0,-1){12}}
\put(28,0){\framebox(6,4){$Pa$}}
\put(34,0){\framebox(6,4){$T$}}
\put(37,4){\line(0,1){2}}
\put(40,0){\framebox(6,4){$Uz$}}
\put(47,2){$C_3$}
\put(43,16){\line(0,-1){12}}
\put(40,16){\framebox(6,4){$Ub$}}
\put(43,20){\line(0,1){2}}
\put(46,16){\framebox(6,4){$V$}}
\put(53,18){$C_2$}
\put(49,20){\line(0,1){2}}
\end{picture}
\end{center}
\end{figure}


                               
                   		
This operation changes the clause $C_3$ to a pure  clause, causing the 
deletion of $C_3$.
   
                  
                 
This example shows an effect which is caused by the reduction rules 
described in this and the following paragraph: Every deletion of a clause 
may cause further deletions of clauses and links (snowball effect).\index{snowball effect}



\PO
\oline{T/Y/YES}{Pure clauses are deleted.}
\oline{$^*$P/PARTIAL}{Removal only if no equations are in the clauses.}
\oline{NIL/N/NO}{Pure clauses remain in the graph.}


\subsubsection{RED.I\_CLAUSE.TAUTOLOGY}
\index{{\sc red.i\_clause.tautology}}
\index{clause tautology} \index{option clause tautology} \index{initial reduction option clause tautology}

This option controls the treatment of tautological clauses. A 
tautological clause is always true under the actual theory. 

\begin{figure}[ht]
\caption{Tautologies}
\label{Tautologies}
\begin{center}
\begin{picture}(24,24)
\put(0,13){a)}
\put(4,4){\framebox(10,4){$Pa$}}
\put(14,4){\framebox(10,4){$\neg Pa$}}
\put(4,20){\framebox(10,4){$Rab$}}
\put(14,20){\framebox(10,4){$\neg Rba$}}
\end{picture}
\qquad
\begin{picture}(30,24)
\put(0,13){b)}
\put(7,4){\framebox(20,4){$ga=ga$}}
\put(4,20){\framebox(26,4){$fafbc=ffabc$}}
\end{picture}
\qquad
\begin{picture}(24,24)
\put(0,13){c)}
\put(0,4){\framebox(8,4){$\neg Px$}}
\put(8,4){\framebox(8,4){$Qb$}}
\put(16,4){\framebox(8,4){$Rx$}}
\put(4,20){\framebox(8,4){$Py$}}
\put(12,20){\framebox(8,4){$\neg Ry$}}
\put(4,8){\line(1,3){4}}
\put(20,8){\line(-1,3){4}}
\put(12,4){\oval(16,8)[b]}
\end{picture}
\end{center}
\end{figure}


\begin{enumerate}
\item A clause containing two literals which have
\begin{itemize}
\item different signs,
\item the same predicate symbol, and
\item equal term lists
\end{itemize}
is a tautology.
Figure \ref{Tautologies}a shows two examples ($R$ is symmetric).


\item  A clause containing a literal of the form $t_1 = t_2$ with $t_1$
equal to $t_2$ is a tautology.
Figure \ref{Tautologies}b shows two examples ($f$ is associative).


\item A clause containing two literals which 
are connected by an R-link with $\epsilon$-unifier caused by the
two-rule algorithm is a tautology.  Figure \ref{Tautologies}c shows an
example.
\end{enumerate}

                            
Additional links created by the two-rule algorithm can also trigger other 
reduction rules. In the following we do not mention this explicitly.

\PO 
\oline{$^*$T/Y/YES}{Removal of initial tautological clauses.}
\oline{NIL/N/NO}{No removal.}

\subsubsection{RED.I\_CLAUSE.TAUTOLOGY.RECHECK}
\index{{\sc red.i\_clause.tautology.recheck}}
\index{clause tautology recheck} \index{option clause tautology recheck} \index{initial reduction option clause tautology recheck}
The tautology reduction rule described so far is incomplete. For example 
suppose we have the three clauses in figure \ref{TautologyLinkCondition}.

\begin{figure}[ht]
\caption{Tautology Link Condition}
\label{TautologyLinkCondition}
\begin{center}
\begin{picture}(38,26)
\put(0,21){$C_1$}
\put(4,20){\framebox(6,4){}}
\put(10,20){\framebox(6,4){$P$}}
\put(10,4){\framebox(6,4){$\neg P$}}
\put(13,8){\line(0,1){12}}
\put(16,4){\framebox(6,4){}}
\put(18,1){$C_3$}
\put(16,22){\line(1,0){6}}
\put(17,23){$L_1$}
\put(22,20){\framebox(6,4){$\neg P$}}
\put(22,4){\framebox(6,4){$P$}}
\put(25,8){\line(0,1){12}}
\put(28,20){\framebox(6,4){}}
\put(35,21){$C_2$}
\end{picture}
\end{center}
\end{figure}
 
                                            
               
If the link $L_1$ is missing, possible deduction steps can be lost by the 
deletion of the tautology $C_3$. The existence of the link $L_1$ is here 
termed as\index{link condition} 
``link condition''. For the reduction rule subsumption a 
similar link condition exists. These link 
conditions make several additional checks necessary: 
\begin{itemize}
\item after link insertion to check if a link condition is now fulfilled
\item after link deletion to check if a link condition becomes now superfluous
\end{itemize}
All recheck options manage additional checks to ensure these conditions
after removals or insertions of links.
The adjustment of the tautology recheck
option controls a renewed tautology check after 
link insertion and deletion. After link deletion the link condition can 
become superfluous.

\Ex

In figure \ref{TautologyLinkCondition} the link condition would be not
fulfilled and the tautology $C_3$ could not be deleted when link $L_1$
was missing. In this case deleting one of the other links would cause
the link condition to become superfluous and $C_3$ could be deleted.
After link insertion a renewed check is also necessary because the
inserted link can fulfill the link condition. If a link is inserted by
the two-rule algorithm, a clause which was no tautology can now become
a tautology.
                                     
\PO
\oline{$^*$PARTIAL/P}{Only clauses which are incident
(i.e. directly connected) to 
the clauses the link is inserted (or deleted) between are 
checked.}
\oline{NIL/N/NO}{The option is switched off.}



\subsubsection{RED.I\_CLAUSE.SUBSUMPTION}
\index{{\sc red.i\_clause.subsumption}}
\index{clause subsumption} \index{option clause subsumption} \index{initial reduction option clause subsumption}

This option controls the treatment of subsumed clauses in the initial
graph. A clause $C$ subsumes a clause $D$ if $C$ has less than or the same
number of literals as
$D$ and there exists a substitution such that $\sigma(C) \subset
D$. If the deduction of the empty clause by $D$ is possible it is also
possible and in most cases even shorter by using the clause $D$. By
this it seems useful to delete the subsumed clause (see also
\cite{Loveland78,Raph84}).

In figure \ref{Subsumption}
$C_1$ can be deleted because it is subsumed by $C_2$. Subsumption
possibilities are checked by using so-called S1-links (see \cite{Eisinger89}).

\begin{figure}[ht]
\caption{Subsumption}
\label{Subsumption}
\begin{center}
\begin{picture}(28,20)
\put(0,1){$C_1$}
\put(0,17){$C_2$}
\put(4,0){\framebox(8,4){$Pz$}}
\put(12,0){\framebox(8,4){$Pfa$}}
\put(20,0){\framebox(8,4){$Sb$}}
\put(4,16){\framebox(8,4){$Px$}}
\put(12,16){\framebox(8,4){$Pfy$}}
\put(8,4){\line(0,1){12}}
\put(16,4){\line(0,1){12}}
\end{picture}
\end{center}
\end{figure}


\PO
\oline{$^*$T/Y/YES}{Subsumed initial clauses are removed.}
\oline{NIL/N/NO}{No removal.}

\subsubsection{RED.I\_CLAUSE.SUBSUMPTION.RECHECK}
\index{{\sc red.i\_clause.subsumption.recheck}}
\index{clause subsumption recheck} \index{option clause subsumption recheck} 
\index{initial reduction option clause subsumption re\-check}

The link condition for the subsumption rule is managed similar to that of
tautology.

                                       
\PO
\oline{$*$PARTIAL/P}{Only clauses which are incident (i.e.\ directly 
connected) to the clauses between which the link is inserted (or deleted)
are checked.}
\oline{NIL/N/NO  }{The option is switched off.}


\subsubsection{RED.I\_REPL.FACTORING}
\index{{\sc red.i\_repl.factoring}}
\index{replacement factoring} \index{option replacement factoring} \index{initial reduction option replacement factoring}

A factor $C$ may subsume its parent clause $D$ and therefore $D$ can 
be deleted. 
Instead of executing the factorization step and then deleting the subsumed 
clause, $C$ can be obtained by simply erasing the appropriate literal of the 
parent clause $D$, i.e.\ factorization and the application of the subsumption 
reduction rule can be grouped together to a macro graph operation. 

\begin{samepage}

\Ex
A factor of $C_1$ in figure \ref{ReplFactoring}a is $C_2$.
$C_1$ is subsumed by $C_2$ and can be deleted. The instantiation and 
erasure of the three 
literals as in figure \ref{ReplFactoring}b has the same effect.
\end{samepage}

\begin{figure}[ht]
\caption{Replacement Factoring}
\label{ReplFactoring}
\begin{center}
\begin{picture}(54,6)
\put(0,1){$C_1$}
\put(0,5){a)}
\put(4,0){\framebox(10,4){$Px$}}
\put(14,0){\framebox(10,4){$Qay$}}
\put(24,0){\framebox(10,4){$Pa$}}
\put(34,0){\framebox(10,4){$Qxy$}}
\put(44,0){\framebox(10,4){$Qaz$}}
\end{picture}
\quad
\begin{picture}(22,6)
\put(0,1){$C_2$}
\put(4,0){\framebox(10,4){$Pa$}}
\put(14,0){\framebox(10,4){$Qaz$}}
\end{picture}\\
\vspace{0.8cm}
\begin{picture}(54,8)
\put(0,3){$C_1$}
\put(0,7){b)}
\put(4,2){\framebox(10,4){$Px$}}
\put(5,0){\line(1,1){8}}
\put(13,0){\line(-1,1){8}}
\put(14,2){\framebox(10,4){$Qay$}}
\put(15,0){\line(1,1){8}}
\put(23,0){\line(-1,1){8}}
\put(24,2){\framebox(10,4){$Pa$}}
\put(34,2){\framebox(10,4){$Qxy$}}
\put(35,0){\line(1,1){8}}
\put(43,0){\line(-1,1){8}}
\put(44,2){\framebox(10,4){$Qaz$}}
\end{picture}
\end{center}
\end{figure}


                                 

A possible generalization is the erasure of literals which become false 
under all theories by the instantiation step as in figure \ref{RFfalse}. 

                                     
\begin{figure}[ht]
\caption{Replacement Factoring with False Literals}
\label{RFfalse}
\begin{center}
\begin{picture}(64,8)
\put(4,2){\framebox(10,4){$Px$}}
\put(5,0){\line(1,1){8}}
\put(13,0){\line(-1,1){8}}
\put(14,2){\framebox(10,4){$Pa$}}
\put(24,2){\framebox(10,4){$Qx$}}
\put(25,0){\line(1,1){8}}
\put(33,0){\line(-1,1){8}}
\put(34,2){\framebox(10,4){$x \neq a$}}
\put(35,0){\line(1,1){8}}
\put(43,0){\line(-1,1){8}}
\put(48,4){$\rightarrow$}
\put(54,2){\framebox(10,4){$Pa$}}
\end{picture}
\end{center}
\end{figure}

%\begin{samepage}

\PO

\oline{$^*$T/Y/YES}{Switched on.}
\oline{NIL/N/NO}{Switched off.}
%\end{samepage}




\subsubsection{RED.I\_REPLACEMENT.FACTORING.RECHECK}
\index{{\sc red.i\_replacement.factoring.re\-check}}
\index{replacement factoring recheck} 
\index{option replacement factoring recheck} 
\index{initial reduction option replacement factoring re\-check}

The adjustment of this option controls a renewed check for replacement 
factoring after link deletion or link insertion. For example suppose the 
situation where a link is inserted by the two-rule algorithm and makes a 
replacement factoring possible. 

\PO
\oline{$^*$T/Y/YES}{Replacement factoring recheck is switched on.}
\oline{NIL/N/NO}{Switched off.}



\subsubsection{RED.I\_CLAUSE.REPL.RESOLUTION}
\index{{\sc red.i\_clause.repl.resolution}}
\index{clause replacement resolution} \index{option clause replacement resolution} \index{initial reduction option clause replacement resolution}

Just as a factor may subsume its parent, a resolvent may subsume one of 
its parents. As above (for RED.I\_CLAUSE.REPL.FACTORING) some 
deduction steps can be 
grouped together to a macro graph operation. 

\begin{enumerate}
\item The simplest case is 
the following:
Suppose we have a unit clause $\{Pa\}$ and a clause  $C_2$: $\{\neg Pa, Qy\}$.
The result of the resolution step $\{Qy\}$ with subsequent subsumption of
the second clause can be 
obtained by simply erasing the literal  $Pa$ in clause $C_2$.

\item One possible generalization is to merge literals: Given clauses
$C_1$: $\{Qb, Pa\}$ and $C_2$: $\{\neg Pa, Qb\}$ one can just erase $Pa$
instead of resolving and subsuming $C_1$.

\item Taking the instantiation process into account we can solve the 
following example by this method:
Resolution between the clauses $C_1$: $\{\neg Px, Pa\}$ and
$C_2$: $\{\neg Pa, Qa\}$ leads to the resolvent
$C_3$: $\{\neg Px, Qa\}$ which subsumes $C_2$. Hence we can achieve the same
effect by replacing the first literal of $C_2$
by the more general $Px$.
\end{enumerate}

A further generalization possibility is the deletion of literals which 
become false in any interpretation by the instantiation process. 

\PO 
\oline{GENERALIZING}{One of the resolution literals can be substituted by a more 
general one of the other clause, as described in 3.}
\oline{$^*$SIMPLE}{A Resolution literal is erased and possible
merging, factoring, and unit-re\-so\-lu\-tion is done (as in 2).}
\oline{UNIT}{One of the parent clauses must be a unit clause (see 1).}
\oline{NIL/N/NO}{Replacement resolution is switched off.}






\subsubsection{RED.I\_CLAUSE.REPL.RESOLUTION.RECHECK}
\index{{\sc red.i\_clause.repl.resolution.re\-check}}
\index{clause replacement resolution recheck} 
\index{option clause replacement resolution recheck} 
\index{initial reduction option clause replacement resolution recheck}

After link deletion or link insertion it is necessary to check a second time 
for replacement resolution. The adjustment of this 
option controls this renewed check for replacement resolution. 
Suppose again that the two-rule algorithm inserts new links.

                                                        
\PO
\oline{$^*$T/Y/YES}{Replacement resolution recheck is switched on.}
\oline{NIL/N/NO}{Switched off.}

\subsubsection{RED.I\_CLAUSE.REWRITING}
\index{{\sc red.i\_clause.rewriting}}
\index{clause rewriting} 
\index{option clause rewriting} 
\index{initial reduction option clause rewriting}

This option controls two different reduction possiblities.
The first one is the application of equations as rewrite rule system.
The second serves as an abbreviation for definitions.
Both can be combined.
To make the axioms better readable (and to save time when typing the 
axioms) it is possible to abbreviate some complex terms. These 
abbreviations are replaced by the original terms during construction of the 
initial graph. This replacement operation is controlled by this option
if set to T or DEF. There are two possibilities

\begin{enumerate}
\item $a = t$     with: $a$ constant, $t$ ground term and $a \not\in t$.
\item $f(x_1, \dots , x_n) = t$  with:   $x_i$ variables with maximal 
domainsort, $f \not\in t$, and $t$ contains no variables $\neq x_i$.

\Ex
\begin{enumerate}
    \item    $sq (x) = times (x\ x)$
    \item   $x \le sq(x)$
will be expanded to   $x \le times (x\ x)$.
\end{enumerate}
\end{enumerate}

Additionally term rewriting is controlled by this option if it is
adjusted to T or DEMODULATION (see section \ref{EqualityReasoning}).
Then all terms are reduced to a normal form according to the
Knuth-Bendix completion algorithm (see the option ER\_COMPLETION).

\PO

\oline{$^*$T/Y/YES}{Clause rewriting is switched on using elimination
with definitions and normalization with ordered equations.}
\oline{DEF}{Elimination of constant and function symbols in initial
clauses using definitions as described above.}
\oline{DEMODULATION}{Normalization with rewrite rules only.}
\oline{NIL/N/NO}{Clause rewriting is switched off.}

\subsubsection{RED.I\_LINK.INCOMPATIBILITY}
\index{{\sc red.i\_link.incompatibility}}
\index{link incompatibility} \index{option link incompatibility} \index{initial reduction option link incompatibility}

An R1-link represents a possible deduction step and is marked with the 
most general unifier of the two literals connected by this R1-link. 
Resolving upon an R1-link makes instantiations of some literals necessary. 
Such an instantiation can block further resolution possibilities. Two links 
are compatible if their unifiers do not contradict each other, i.e.\
resolution upon one link does not block resolving on the other link. Some 
links are simultaneously compatible if their substitutions don't contradict 
each other. A link can be deleted as incompatible, if not every other literal 
has at least one link so that they are simultaneously compatible.

\PO
\oline{T/Y/YES}{Incompatible links are deleted.}
\oline{$^*$PARTIAL}{Only one step purity look-ahead.}
\oline{NIL/N/NO}{No removal of incompatible links.}

\paragraph{Remark} 
The check for link incompatibility is often very expensive and only a
few links are removed. Therefore it seems very useful to choose only
the ``partial''-adjustment or even to switch this option off.



\subsubsection{RED.I\_LINK.TAUTOLOGY}
\index{{\sc red.i\_link.tautology}}
\index{link tautology} \index{option link tautology} \index{initial reduction option link tautology}

The resolvent of two clauses can be a tautology. Tautologies can be 
removed from the graph. Therefore it seems useful to remove links which 
would generate tautologies.

For the handling of the link condition we introduce a state ``inhibited''
for links, which can not be removed. Inhibited links are not selected for
any operation until no other links are available. This state is not inherited.

\PO
\oline{T/Y/YES}{Removal of tautology-generating links without link
condition  check.}
\oline{$^*$REMOVE-INHIBIT/RI}%
{Removal of the links complying with the link condition.
Inhibition of the others.}
\oline{INHIBIT/I}{Inhibition of tautology-generating links.}
\oline{REMOVE/R}{Removal of the links complying with the link condition.}
\oline{NIL/N/NO}{No removal, no inhibition.}

\subsubsection{RED.I\_LINK.TAUTOLOGY.RECHECK}
\index{{\sc red.i\_link.tautology.recheck}}
\index{link tautology recheck} \index{option link tautology recheck} \index{initial reduction option link tautology recheck}

The adjustment of this option controls a renewed check for a link
tautology after link insertion or link removal.

\PO
\oline{T/Y/YES}{The option is switched on.}
\oline{$^*$NIL/N/NO}{switched off.}  

\subsubsection{RED.I\_LINK.SUBSUMPTION}
\index{{\sc red.i\_link.subsumption}}
\index{link subsumption} \index{option link subsumption} \index{initial reduction option link subsumption}

Links which generate clauses that are subsumed by other clauses can be 
removed. This removal operation is controlled by this option.


\PO
\oline{T/Y/YES}{Removal of the links without link condition check.}
\oline{$^*$REMOVE-INHIBIT/RI}{Removal of the links complying with the
link condition. Inhibition of the others.}
\oline{INHIBIT/I}{Inhibition of the links.}
\oline{REMOVE/R}{Removal of the links complying with the link 
condition.}
\oline{NIL/N/NO}{No removal, no inhibition.}

\subsubsection{RED.I\_LINK.SUBSUMPTION.RECHECK}
\index{{\sc red.i\_link.subsumption.recheck}}
\index{link subsumption recheck} \index{option link subsumption recheck} \index{initial reduction option link subsumption recheck}

This option influences a renewed link subsumption check upon link removal 
or link insertion.

\PO
\oline{T/Y/YES}{Switched on.}
\oline{$^*$NIL/N/NO}{Switched off.}


\subsection{Options of the Category RED.D}

This option-area consists of options for various reduction rules during 
deduction.

\subsubsection{RED.D\_CLAUSE.MULTIPLE.LITERALS}
\index{{\sc red.d\_clause.multiple.literals}}
\index{clause multiple literals} \index{option clause multiple literals} \index{deduction reduction option clause multiple literals}

This option has the same effect as RED.I\_CLAUSE.MULTIPLE.LITERALS.

\PO
\oline{$^*$T/Y/YES}{Option is switched on.}
\oline{NIL/N/NO}{Option is switched off.}

\subsubsection{RED.D\_CLAUSE.PURITY}
\index{{\sc red.d\_clause.purity}}
\index{clause purity} \index{option clause purity} \index{deduction reduction option clause purity}

The option has the same effect as RED.I\_CLAUSE.PURITY.

\PO

\oline{T/Y/YES}{Pure clauses are deleted.}
\oline{$^*$P/PARTIAL}{Removal only if no equations are in the clauses.}
\oline{NIL/N/NO}{Pure clauses remain in the graph.}



\subsubsection{RED.D\_CLAUSE.TAUTOLOGY}
\index{{\sc red.d\_clause.tautology}}
\index{clause tautology} \index{option clause tautology} \index{deduction reduction option clause tautology}

Treatment of deduced tautology clauses.
Such clauses can be removed if no deduction possibility is lost by the 
removal (for precise explanation see RED.I\_CLAUSE.TAUTOLOGY).

\PO
\oline{T/Y/YES}{Removal of the tautology clauses 
without link                                                    condition check.}
\oline{$^*$REMOVE-INHIBIT/RI}{Removal of the clauses where link condition is 
met, reinsertion and inhibition of creator links.}
\oline{INHIBIT/I}{Removal of the clauses as well as reinsertion and
inhibition of creator links.}
\oline{REMOVE/R}{Removal of the links complying with the link 
condition.}
\oline{NIL/N/NO}{No removal, no inhibition.}

\subsubsection{RED.D\_CLAUSE.TAUTOLOGY.RECHECK}
\index{{\sc red.d\_clause.tautology.recheck}}
\index{clause tautology recheck} \index{option clause tautology recheck} 
\index{deduction reduction option clause tautology re\-check}

Renewed tautology check and treatment after insertion or removal of links 
(for precise explanation see RED.I\_CLAUSE.TAUTOLOGY.RECHECK).


\PO
\oline{T/Y/YES}{Switched on for insertion and removal.}
\oline{PARTIAL/P}{Switched on for insertion of incident links.}
\oline{$^*$NIL}{Switched off.}


\subsubsection{RED.D\_CLAUSE.SUBSUMPTION.FORWARD}
\index{{\sc red.d\_clause.subsumption.for\-ward}}
\index{clause subsumption forward} \index{option clause subsumption forward} \index{deduction reduction option clause subsumption forward}

Treatment of deduced clauses which are subsumed by previous clauses.

\PO
\oline{T/Y/YES}{ Removal of the clauses without link condition check.}
\oline{$^*$REMOVE-INHIBIT/RI}{In case of a deduced subsumed clause, removal of 
the clause. If link condition is met reinsertion and                      
inhibitation  of creator link.}
\oline{INHIBIT/I}{Removal of the clauses as well as reinsertion and 
inhibition                                of creator links.}
\oline{REMOVE/R}{Removal of the clauses complying with the link
condition.}
\oline{NIL/N/NO}{No removal, no inhibition.}

\subsubsection{RED.D\_CLAUSE.SUBSUMPTION.BACKWARD}
\index{{\sc red.d\_clause.subsumption.back\-ward}}
\index{clause subsumption backward} \index{option clause subsumption backward} \index{deduction reduction option clause subsumption backward}

Treatment of clauses that are subsumed by deduced new clauses.


\PO
\oline{T/Y/YES}{Removal of all appropriate clauses without link condition 	check.}
\oline{$^*$REMOVE}{Only removal of such clauses which comply with the link 	condition.}
\oline{NIL/N/NO}{No removal.}


\subsubsection{RED.D\_CLAUSE.SUBSUMPTION.RECHECK}
\index{{\sc red.d\_clause.subsumption.re\-check}}
\index{clause subsumption recheck} 
\index{option clause subsumption recheck} 
\index{deduction reduction option clause subsumption re\-check}

This option controls a renewed subsumption check, if subsumptions 
become possible or no longer possible by link insertion or 
removal, respectively.

\PO
\oline{T/Y/YES}{Switched on for insertion and removal of links.}
\oline{$^*$REMOVE/R}{Removal of the clauses complying with the link condition.}
\oline{NIL/N/NO}{The option is switched off.}

\subsubsection{RED.D\_CLAUSE.REPL.FACTORING}
\index{{\sc red.d\_clause.repl.factoring}}
\index{clause replacement factoring} \index{option clause replacement factoring} \index{deduction reduction option clause replacement factoring}

The option has the same effect as RED.I\_CLAUSE.REPL.FACTORING.

\PO
\oline{$^*$T/YES/Y}{Switched on.}
\oline{NIL/N/NO}{Switched off.}

\subsubsection{RED.D\_CLAUSE.REPL.FACTORING.RECHECK}
\index{{\sc red.d\_clause.repl.factoring.re\-check}}
\index{clause replacement factoring recheck} 
\index{option clause replacement factoring recheck} 
\index{deduction reduction option clause replacement factoring re\-check}

The option has the same effect as RED.I\_CLAUSE.REPL.FACTORING.RECHECK.

\PO
\oline{$^*$T/Y/YES}{Switched on.}
\oline{NIL/N/NO}{Switched off.}

\subsubsection{RED.D\_CLAUSE.REPL.RESOLUTION}
\index{{\sc red.d\_clause.repl.resolution}}
\index{clause replacement resolution} \index{option clause replacement resolution} \index{deduction reduction option clause replacement resolution}

The option has the same effect as RED.I\_CLAUSE.REPL.RESOLUTION

\PO
\oline{GENERALIZING/G}{One of the resolution literals is replaced by a more 
general                                 one of the other clause.}
\oline{$^*$SIMPLE/S}{Switched on, only for one step, without generalizing.}
\oline{UNIT/U}{Switched on for unit partner only.}
\oline{NIL/N/NO}{Switched off.}

\subsubsection{RED.D\_CLAUSE.REPL.RESOLUTION.RECHECK}
\index{{\sc red.d\_clause.repl.resolution.re\-check}}
\index{clause replacement resolution recheck} \index{option clause replacement resolution recheck} \index{deduction reduction option clause replacement resolution recheck}

The option has the same effect as RED.I\_CLAUSE.REPL.RESOLUTION.RECHECK.

\PO
\oline{$^*$T/YES/Y}{Switched on.}
\oline{NIL/N/NO}{Switched off.}

\subsubsection{RED.D\_CLAUSE.REWRITING}
\index{{\sc red.d\_clause.rewriting}}
\index{clause rewriting} \index{option clause rewriting} \index{deduction reduction option clause rewriting}

See RED.I\_CLAUSE.REWRITING.
\PO

\oline{$^*$T/Y/YES}{Clause rewriting is switched on using elimination
with definitions and normalization with ordered equations.}
\oline{DEF}{Elimination of constant and function symbols in initial
clauses using definitions as described above.}
\oline{DEMODULATION}{Normalization with rewrite rules only.}
\oline{NIL/N/NO}{Clause rewriting is switched off.}



\subsubsection{RED.D\_LINK.INCOMPATIBILITY}
\index{{\sc red.d\_link.incompatibility}}
\index{link incompatibility} \index{option link incompatibility} \index{deduction reduction option link incompatibility}

The option has the same effect as RED.I\_LINK.INCOMPATIBILITY.

\PO
\oline{$^*$T/YES/Y}{Removal of the links.}
\oline{NIL/N/NO}{No removal.}

\subsubsection{RED.D\_LINK.TAUTOLOGY}
\index{{\sc red.d\_link.tautology}}
\index{link tautology} \index{option link tautology} 
\index{deduction reduction option link tautology}

The option has the same effect as RED.I\_LINK.TAUTOLOGY.

\PO
\oline{T/Y/YES}{Removal of the links without link condition check.}
\oline{$^*$REMOVE-INHIBIT/RI}{Removal of the links complying with the link 
condition. 	           Inhibition of the others.}
\oline{INHIBIT/I}{Inhibition of the links.}
\oline{REMOVE/R}{Removal of the links complying with the link condition.}
\oline{NIL/N/NO}{No removal, no inhibition.}



\subsubsection{RED.D\_LINK.TAUTOLOGY.RECHECK}
\index{{\sc red.d\_link.tautology.recheck}}
\index{link tautology recheck} \index{option link tautology recheck} 
\index{deduction reduction option link tautology re\-check}

The option has the same effect as RED.I\_LINK.TAUTOLOGY.RECHECK.

\PO
\oline{T/Y/YES}{Switched on for removal and insertion.}
\oline{PARTIAL/P}{Switched on for insertion of adjacent links.}
\oline{$^*$NIL/N/NO}{Switched off.}

\subsubsection{RED.D\_LINK.SUBSUMPTION}
\index{{\sc red.d\_link.subsumption}}
\index{link subsumption} \index{option link subsumption} \index{deduction reduction option link subsumption}

The option has the same effect as RED.I\_LINK.SUBSUMPTION.

\PO
\oline{T/YES/Y}{Removal of links without link condition check.}
\oline{$^*$REMOVE-INHIBIT/RI}{Removal of the links complying with the link 
condition. Inhibition of the others.}
\oline{INHIBIT/I}{Inhibition of the links.}
\oline{REMOVE/R}{Removal of the links complying with the link condition.}
\oline{NIL/N/NO}{No removal, no inhibition.}

\subsubsection{RED.D\_LINK.SUBSUMPTION.RECHECK}
\index{{\sc red.d\_link.subsumption.recheck}}
\index{link subsumption recheck} \index{option link subsumption recheck} 
\index{deduction reduction option link subsumption re\-check}

The option has the same effect as RED.I\_LINK.SUBSUMPTION.RECHECK.

\PO
\oline{T/Y/YES}{Switched on for removal and insertion.}
\oline{PARTIAL/P}{Switched on for insertion of adjacent links.}
\oline{$^*$NIL/N/NO}{Switched off.}




\subsection{Options of the Category STR}

This area consists of options, which direct the search-behaviour while 
looking for a proof.

\subsubsection{FAC\_INITIAL}
\index{{\sc fac\_initial}}
\index{initial} \index{option initial} \index{fac option initial}

This option controls factorizations in the initial graph.

\PO
\oline{T/Y/YES}{The option is switched on.}
\oline{$^*$NIL/N/NO}{The option is switched off.}

\subsubsection{FAC\_EACH.STEP}
\index{{\sc fac\_each.step}}
\index{each step} \index{option each step} \index{fac option each step}

Factorizing after each deduction step.

\PO
\oline{T/Y/YES}{The option is switched on.}
\oline{$^*$NIL/N/NO}{The option is switched off.}

\subsubsection{STR\_RESOLUTION}
\index{{\sc str\_resolution}}
\index{resolution} \index{option resolution} \index{strategy option resolution}

Adjustment of the basic deduction strategy during proof search. Various 
classical refinement strategies like set-of-support, linear etc.\ are 
available and are simulated in the connection graph calculus by marking
R-links as ``active'' or ``passive''.

\PO
\olineeins{BASIC-RESOLUTION/BASIC/B}
\oline{}{All links are marked as active such that 
the 		selection module does a breadth first 		search.}
\olineeins{$^*$SET-OF-SUPPORT/SOS/S}
\oline{}{Only if there is at least one theorem 
clause 		the Set-of-Support strategy will be 
applied, 		else the strategy is switched to basic. }
\olineeins{UNIT-REFUTATION/UNIT/U}
\oline{}{All links connected to unit clauses are activated, the other 
ones are marked passive.}
\olineeins{LINEAR/L/LINEAR.AXM\#/LINEAR.THM\#/L.AXM\#/L.THM\#}
\oline{\quad}{The first clause the linear strategy 
starts with (i.e. the ``top clause'') is either 
defined by using a concatenation of ``linear.axm'' or ``linear.thm''  and 
the printname of the top clause as the 
strategy name, or is asked for by the system if one 
uses STR\_RESOLUTION = LINEAR (in this case 
the system prints all clauses with their printname).}

Combined strategies: 	

U-B/Unit-Basic

U-SOS/Unit-Set-of-Support

U-L/U-L.AXM\#/U-L.THM\#-LINEAR
 
Unit Resolution prunes the search space considerably, but
unfortunately it is not complete for all clause sets. The user can
define strategies: U-SOS, U-B, and U-L saying 	``if
the clause set is horn renamable the unit-resolution, else
Set-of-Support (basic, or linear strategy, respectively)''.



\subsubsection{STR\_LINK.DEPTH}
\index{{\sc str\_link.depth}}
\index{link depth} \index{option link depth} \index{strategy option link depth}

This option restricts the depth of the search space for the empty
clause (with respect to the initial clauses).
This link depth is defined as follows:

	Links between clauses in the initial graph have the link 
depth 	$d(L) = 0$.
	For all other $L$: $d(L) = 1 + max (d(L_1), d(L_2))$ where $L_1$ and 
$L_2$ 	are the links which generated the two clauses connected 
by $L$.

\PO
\oline{Positive integer}{Upper bound for link depth.}
\oline{$^*$N/NO/NIL}{No upper bound.}


\subsubsection{STR\_R.SELECTION}
\label{strrselection}
\index{{\sc str\_r.selection}}
\index{r selection} \index{option r selection} \index{strategy option r selection}

The value of this option determines the selection function for R-links
in the H-C strategy of ER\_PA\-RA\-MO\-DU\-LA\-TION. The smallest link relative to this function
is selected. The value must be positive.

The argumentlist of the selection function is
(WEIGHT NOLIT VARIABLES DEPTH SUPPORT EQUATIONAL).

WEIGHT is the weight of the link, that is, a value computed from the
potential result. It is determined by the heuristic value function
defined by STR\_WEIGHT.POLYNOMIALS. WEIGHT is always 0 for R-links 
if ER\_PARAMODULATION is switched to HEURISTIC-COMPLETION. 

NOLIT is the number of literals in the result clause.

VARIABLES is the potential number of variables in the result clause.

DEPTH is the depth of the link in the search space.

SUPPORT is true iff one of
the parents is in the set of support.

EQUATIONAL is true iff all literals in the clause are positive
equations. For this this option EQUATIONAL is always true.

\PO  
\olineeins{A Lisp expression using the six parameters above as free variables.}
\oline{Default Value}{(* 10 (+ 2 VARIABLES (* 2 DEPTH) (* 3 NOLIT)))}  


\subsubsection{STR\_TERM.DEPTH}
\index{{\sc str\_term.depth}}
\index{term depth} \index{option term depth} \index{strategy option term depth}

By this option an upper bound for the term nesting depth can be specified. 
This prevents deducing terms with depth greater than the specified one 
and thereby restricting  the search-space. It is defined as $d(c) = 0$
for constants, $d(x) = 0$ for variables, and $d(f(t_1,\dots ,t_n)) =
1+ max(d(t_1),\dots ,d(t_n))$.

\PO
\oline{Positive integer}{Upper bound for term nesting depth.}
\oline{$^*$N/NO/NIL}{No upper bound.}

\subsubsection{STR\_FINITE.DOMAIN}
\index{{\sc str\_finite.domain} \index{finite domain} }
\index{option finite domain}   \index{strategy option finite domain}

\PO
\oline{$^*$T/Y/YES}{Special way of handling finite domains,
           which are defined via clauses X=A1, X=A2, ..., X=AN, by
instantiating all derived clauses (possibly incomplete).}
\oline{N/NIL/NO}{No special handling.}


\subsubsection{TERM\_UNITS}
\index{{\sc term\_units}}
\index{units} \index{option units} \index{terminator option units}
\label{termunits}

The option controls processing of proof patterns leading to unit
clauses found by the terminator.

\PO

\oline{$^*$T/Y/YES}{The resolution steps necessary to produce the unit clauses 
proposed by the terminator are performed.}
\oline{NIL/N/NO}{Resolution possibilities leading to unit clauses found by 
the 	terminator are ignored.}

\subsubsection{TERM\_ITERATIONS}
\index{{\sc term\_iterations}}
\index{iterations} \index{option iterations} 
\index{terminator option iterations}\label{termiterations}

This option influences the length of a path which will be processed by
the terminator proceeding from each clause to detect a ``terminator
situation''. Terminator situations are like the one displayed in
figure \ref{TerminatorSituation}. This connection graph \index{terminator situation}
has no cycles in contrast to
figure \ref{ConnectionGraph}. They are searched beginning with unit
clauses (see \cite{AnOh83}).

\begin{figure}[ht]
\caption{Terminator Situation}
\label{TerminatorSituation}
\begin{center}
\begin{picture}(56,40)
% Oberste Zeile
\put(0,33){\framebox(6,4){}}
\put(18,33){\framebox(6,4){}}
\put(24,33){\framebox(6,4){}}
\put(30,33){\framebox(6,4){}}
\put(42,33){\framebox(6,4){}}
\put(48,33){\framebox(6,4){}}
% Links zwischen erster und zweiter Zeile
\put(3,33){\line(0,-1){6}}
\put(21,33){\line(-2,-1){12}}
\put(27,33){\line(0,-1){6}}
\put(36,35){\line(1,0){6}}
\put(51,33){\line(-2,-1){12}}
% 2. Zeile
\put(0,23){\framebox(6,4){}}
\put(6,23){\framebox(6,4){}}
\put(12,23){\framebox(6,4){}}
\put(24,23){\framebox(6,4){}}
\put(36,23){\framebox(6,4){}}
% Links zw 2. und 3. Zeile
\put(15,23){\line(0,-1){6}}
% 3. Zeile
\put(0,13){\framebox(6,4){}}
\put(6,13){\framebox(6,4){}}
\put(12,13){\framebox(6,4){}}
% Links zwischen 3. und 4. Zeile
\put(3,13){\line(0,-1){6}}
\put(9,13){\line(1,-1){6}}
% Letzte Zeile
\put(0,3){\framebox(6,4){}}
\put(12,3){\framebox(6,4){}}
\end{picture}
\end{center}
\end{figure}


\PO
\oline{$^*$0}{No new unit clauses are generated and only the one level 	terminator situations can be found.}
\oline{integer $>$ 0}{Deeper level terminator situations can be found, but more
	time consuming.}


\paragraph{Remark}	The number of TERM\_ITERATIONS should not be chosen too high 	(normally 3-5) because the time for one deduction step 	increases explosively. However this option can be 	coordinated with 
STR\_TERM.DEPTH to restrict the set of clauses to 	examine.
 	
\subsubsection{TERM\_SET.OF.SUPPORT}
\index{{\sc term\_set.of.support}}
\index{set of support} \index{option set of support} \index{terminator option set of support}

Allows restricting the set of unit clauses examined by the terminator
to detect a terminator situation.

\PO
\oline{$^*$NIL/N/NO}{No restriction of the unit clauses.}
\oline{T/Y/YES}{The terminator only uses unit clauses of the set of support.}

\subsubsection{TERM\_BREADTH.FIRST}
\index{{\sc term\_breadth.first}}
\index{breadth first} \index{option breadth first} \index{terminator option breadth first}

Allows switching the search strategy of the terminator to breadth
first.

\PO
\oline{T/Y/YES}{Pure breadth first search.}

\oline{$^*$NIL/N/NO}{The basic search strategy of the terminator is used 
(similar to 		the usual linear strategies).}


\subsection{Options of the Category SORT}

These options control the behaviour of the unifier 
construction for a version of {\sc Mkrp} with dynamic sorts.

The unification problem for dynamic sorts is in general undecidable and
of type {\em infinitary\/}. Therefore it is necessary to restrict the search
tree of the unification algorithm. This is controlled by the options
SORT\_MAX.UNIFICATION.RULE.STEPS, SORT\_MAX.UNIFICATION.TREE.DEPTH,
SORT\_MAX.UNIFICATION.TREE.OPEN.NODES, and SORT\_UNIFIER.STOP.NUMBER. If
the algorithm stops due to the restriction imposed by one of the
parameters, the unsolved subproblems are added to the resolvent as
residues. Hence the completeness of the calculus is preserved.

\subsubsection{SORT\_LITERALS}
\index{{\sc sort\_literals}}
\index{literals} \index{option literals} \index{sort option literals}

\PO                      
\oline{T/Y/YES}{Special way of handling sorts using sort literals (see
\cite{Weidenbach91,Weidenbach93}).}
\oline{$^*$N/NIL/NO}{Normal sorts.}

\subsubsection{SORT\_MAX.UNIFICATION.RULE.STEPS}
\index{{\sc sort\_max.unification.rule.steps}}
\index{max unification rule steps} \index{option max unification rule steps} 
\index{sort option max unification rule steps}

\PO
\olineeins{$^*$100}
\oline{integer $>$ 0}{Upper bound for the number of rule steps to
                be performed.}


\subsubsection{SORT\_MAX.UNIFICATION.TREE.DEPTH}
\index{{\sc sort\_max.unification.tree.depth}}
\index{max unification tree depth} \index{option max unification tree depth} 
\index{sort option max unification tree depth}

\PO
\olineeins{$^*$100}
\oline{integer $>$ 0}{Upper bound for the depth of the search tree
                in the unification algorithm.}


\pagebreak[4]
\subsubsection{SORT\_MAX.UNIFICATION.TREE.OPEN.NODES}
\index{{\sc sort\_max.unification.tree.open.nodes}}
\index{max unification tree open nodes} \index{option max unification tree open nodes}
\index{sort option max unification tree open nodes}

\PO
\olineeins{$^*$20}
\oline{integer $>$ 0}{Upper bound for the leaf number of the search tree
                in the unification algorithm.}


\subsubsection{SORT\_UNIFIER.STOP.NUMBER}
\index{{\sc sort\_unifier.stop.number}}
\index{unifier stop number} \index{option unifier stop number} 
\index{sort option unifier stop number}

\PO
\olineeins{$^*$100}
\oline{integer $>$ 0}{Upper bound for the number of most general unifiers to
                be searched.}

\subsubsection{SORT\_SHOW.VARIABLE.SORTS}
\index{{\sc sort\_show.variable.sorts}}
\index{show variable sorts} \index{option show variable sorts} 
\index{sort option show variable sorts}

\PO                      
\oline{$^*$T/Y/YES}{When SORT\_LITERALS is switched on all Variables are printed
           together with their sorts.}
\oline{N/NIL/NO}{Normal output.}


\subsection{Options of the Category ER}

These are the options to control the built in equality resoning
procedure. For the general procedure see section \ref{EqualityReasoning}.




\subsubsection{ER\_PARAMODULATION}
\index{{\sc er\_paramodulation}}
\index{paramodulation} \index{option paramodulation}
\index{equality option paramodulation}\label{erparamodulation}

Choosing the paramodulation-strategy:

\newdimen\il
\newlength{\rulewithif}
\newlength{\rulewithoutif}
\newlength{\iflength}
\newlength{\rulerestwidth}
\newlength{\malength}
\newbox\block
\newbox\blocka
\newbox\blockb


\def\hlineblocktwo#1#2#3#4{%
  \setbox\blocka=\hbox{#2}
  \setbox\blockb=\hbox{#3}
  \ifdim\wd\blocka<\wd\blockb\il=\wd\blockb\else\il=\wd\blocka\fi
  \setbox\blockb=\hbox{#4}
  \ifdim\il<\wd\blockb\il=\wd\blockb\else\fi
  {\tolerance=2000
  \spaceskip=3.33pt plus 10pt minus 1.11pt
  \global\setbox\block=\vbox{\hsize=\il #3\\#4 \hrule  #2}}}

\def\inferencerule#1{% Block in box \block   1: condition
  \tolerance=2000
  \spaceskip=3.33pt plus 10pt minus 1.11pt
  \topsep0cm
  \itemsep0cm
  \parskip0mm
  \parsep1mm
  \setlength{\rulewithoutif}{\wd\block}
  \settowidth{\iflength}{\quad if \quad}
  \setlength{\rulewithif}{\rulewithoutif}
  \addtolength{\rulewithif}{\iflength}
  \setlength{\rulerestwidth}{\textwidth}
  \addtolength{\rulerestwidth}{-1\rulewithif}  % Verbleibende Textbreite
  \setbox\blocka=\vbox{\parbox{\rulerestwidth}{\raggedright #1}}
  %\ifdim\ht\block>\ht\blocka%
  %\else{%
  % \centerline{\hbox{\box\block}}
  % \centerline{if}
  % \begin{center}\begin{minipage}{10cm}#1\end{minipage}\end{center}}%
  %\fi
}



\olineeins{$^*$HEURISTIC-COMPLETION}
\oline{}{Paramodulation steps are used to generate new clauses from
critical pairs. The resolution strategy STR\_RESOLUTION is followed
for resolutions, that is, the resolution part of the theorem prover
behaves as if no equality is present. All other strategies change the
resolution behaviour, then it is controlled by the option
ER\_P.SELECTION instead of STR\_R.SELECTION.}
\oline{CLAUSE-GRAPH}{Ordered resolution and paramodulation with inheritance 
of R-links.}
\oline{ZHANG-KAPUR}{Strategy corresponding to \cite{ZhKa88}}
{\samepage
\olineeins{BACHMAIR-GANZINGER}
\oline{}{Strategy corresponding to \cite{BaGa90}.}}
\oline{SNYDER-LYNCH}{Basic paramodulation strategy, like basic narrowing \cite{SnLy91}.}
\oline{DERSHOWITZ}{Unit-strategy \cite{Dershowitz91} if Horn clauses, else incomplete.}
\pagebreak[3]


For a brief introduction into the history of the superposition
approach see section \ref{EqualityReasoning}.  The ordering is
extended from the usual reduction ordering of the Knuth-Bendix
algorithm to the literals in the clauses. For the following
definitions we call a Literal $L$ {\em maximal in} a set of literals
$\{L_1$, $\dots$, $L_n\}$ iff for all $i$ $L_i \not> L$, we call it
strictly maximal if $L_i \not> L$ and $L_i \not= L$.



The inference rule scheme for the Z-K strategy is:


\hlineblocktwo%
{\ }%
{$\sigma(s[u\leftarrow r] = t$, $L_1$, $\dots$, $L_n$, $K_1$,
$\dots$, $K_m)$}%
{$s = t$, $L_1$, $\dots$, $L_n$}%
{$l = r$, $K_1$, $\dots$, $K_m$}
\inferencerule{%
$u$ is a non variable position of $s$, $\sigma$ is an mgu of $s|u$ and $l$, $s=t$ 
is maximal, { $l=r$ is maximal}, {$t \not> s$}, and $\sigma(r) \not> \sigma(l)$.}


This inference rule is incomplete together with the tautology removal
rule, as L.\ Bachmair and H.\ Ganzinger argue. The repaired the
calculus in the following way: Two rules must be defined. The first
one is strict superposition:


\hlineblocktwo%
{+}%
{$\sigma(sign(s[u\leftarrow r] = t)$, $L_1$, $\dots$, $L_n$, $K_1$,
$\dots$, $K_m)$}%
{$sign(s = t)$, $L_1$, $\dots$, $L_n$}%
{$(l = r)$, $K_1$, $\dots$, $K_m$}
\inferencerule{
\begin{itemize}
\topsep0cm
\itemsep0cm
\tabcolsep1mm
\item $u$ is a non variable position of $s$, 
$\sigma$ is an mgu of $s|u$ and $l$.
\item $\sigma(r) \not> \sigma(l)$ 
\item $\sigma(l = r)$ strictly maximal in $\sigma(K_1$, $\dots$, $K_m)$
\item $\sigma(l)$ does not occur in the negative literals of $\sigma(K_1$, $\dots$, $K_m)$
\item $\sigma(t) \not> \sigma(s)$
\item $\sigma(sign(s = t))$ is strictly maximal in all positive literals of
$\sigma(L_1$, $\dots$, $L_n)$. If $sign$ is positive it is also strictly maximal for all
negative literals, else it is only maximal
\end{itemize}}


The second one is named merging paramodulation:


\hlineblocktwo%
{+}%
{$\sigma(s = t[u\leftarrow r]$, $s = t'$, $L_1$, $\dots$, $L_n$, $K_1$,
$\dots$, $K_m)$}%
{$(s = t)$, $(s' = t')$, $L_1$, $\dots$, $L_n$}%
{$(l = r)$, $K_1$, $\dots$, $K_m$}
\inferencerule{%
\topsep0cm
\itemsep0cm
\tabcolsep1mm
\begin{itemize}
\item $\sigma = \tau\rho$ where $\tau$ is an mgu of $t|u$ and $l$, $\rho$ an
      mgu of $\tau(s)$ and $\tau(s')$, and $u$ is a non variable position of $t$.
\item $\sigma(r) \not> \sigma(l)$.
\item $\sigma(l = r)$ strictly maximal in $\sigma(K_1$, $\dots$, $K_m)$
\item $\sigma(l)$ does not occur in the negative literals of $\sigma(K_1$, $\dots$, $K_m)$
\item $\sigma(s) \not> \sigma(t)$.
\item $\sigma(s = t)$ strictly maximal in $\sigma(L_1$, $\dots$, $L_n)$
\item $\sigma(s)$ does not occur in the negative literals of $\sigma(L_1$, $\dots$, $L_n)$
\item $\tau(s) > \tau(t)$ and $\sigma(t') \not\geq \sigma(t)$.
\end{itemize}}


Of course for all strategies a factoring rule on maximal literals has
to be added. We omit the rules for the other strategies because they have
more intuitive names.

\pagebreak[3]

\subsubsection{ER\_COMPLETION }
\index{{\sc er\_completion }}
\index{completion } \index{option completion } 
\index{equality option completion}\label{ercompletion}

  Selection of completion strategy:

\oline{IGNORING}{If undirectable equations occur they are ignored.}
\oline{FAILING}{If undirectable equations occur the proof is aborted.}
\oline{$^*$UNFAILING}{If undirectable equations occur they are used in both
               directions and demodulation can be done with instances
of them.}
\olineeins{CONSTANT-CONGRUENCE}
\oline{\quad}{If undirectable equations occur they are reduced considering
variables as constants.}

\subsubsection{ER\_WEIGHT.POLYNOMIALS}
\label{erweightpolynomials}
\index{{\sc er\_weight.polynomials}}
\index{weight polynomials} \index{option weight polynomials} \index{equality option weight polynomials}

Specification of the polynomials for the weighting of function symbols and
the values for the constants in form of an associationlist.

This weighting is used as WEIGHT in STR\_R.SELECTION and ER\_P.SELECTION
in all strategies and for all links but not for
R-links in HEURISTIC-COMPLETION, there the heuristic value is always 0 (compatibility reasons).
The most similar value for ER\_PARAMODULATION to use WEIGHT would be CLAUSE-GRAPH.

Syntax of polynomials:

\begin{tabbing}		
atom \quad \= ::= \quad \= \ \kill		    
pol \> ::= \> (+ mon ... mon) {\tt |} mon\\
mon \> ::= \> (* atom ... atom) {\tt |} atom\\
atom \> ::= \> $<$number$>$ {\tt |} $<$variable$>$\\
$<$number$>$ \>\> is positive integer or zero.\\
$<$variable$>$ \>\> is one of the argument variables.
\end{tabbing}

\PO
\olineeins{An association list associating polynomial to names.}
\oline{Default value}{()}

\Ex
\begin{tabbing}
(\=(P (X Y) (+ 3 X Y))\\
\> (f (X Y) (+ X (* X Y)))\\ 
\> (inv (X) (* X X))
\\ \> (zero () 1)\\ 
\> (id () 2))
\end{tabbing}

The heuristic value of {\tt P(f(zero id) id)} would be 8.


\subsubsection{ER\_P.SELECTION}
\label{erpselection}
\index{{\sc er\_p.selection}}
\index{p selection} \index{option p selection} \index{equality option p selection}

The value of this option determines the selection function for R-links
in all strategies of ER\_PA\-RA\-MO\-DU\-LA\-TION but
HEURISTIC-COMPLETION and the selection function for P-links in all
strategies. The smallest link relative to this function is selected. The
value must be positive.

The argumentlist of the function is
(WEIGHT NOLIT VARIABLES DEPTH SUPPORT EQUA\-fTIO\-NAL).

WEIGHT is the weight of the link, that is, a value computed from the
potential result. 
It is determined by the heuristic value function
defined by STR\_WEIGHT.POLYNOMIALS.
WEIGHT is always 0 for R-links if ER\_PARAMODULATION is switched
to HEURISTIC-COMPLETION.

NOLIT is the number of literals in the result clause.

VARIABLES is the potential number of variables in the result clause.
For this option VARIABLES is always 1.

DEPTH is the depth of the link in the search space.

SUPPORT is true iff one of the parents is in the set of support.

EQUATIONAL is true iff all literals in the clause are positive
equations.

\PO  
\olineeins{A Lisp expression using the six parameters above as free variables.}
\oline{Default Value}{(* WEIGHT (if SUPPORT 1 1.5) (if EQUATIONAL 1 2))}  

\subsubsection{ER\_CP.REDUCTION}
\index{{\sc er\_cp.reduction}}
\index{cp reduction} \index{option cp reduction} \index{equality option cp reduction}

\oline{NIL/N/NO}{Critical pairs are not reduced, that is, no preview on the
rewriting of links.}
\oline{PARTIAL/P}{Critical pairs are just reduced if they are constructed.}
\oline{$^*$T/Y/YES}{Critical pairs that is links are kept always reduced.
In the case of HEURISTIC-COMPLETION only P1- and P2-links are considered, in
the others R2-links are considered too.}


\subsubsection{ER\_ORDERING}
\index{{\sc er\_ordering}}
\index{ordering} \index{option ordering} \index{equality option ordering}

Selection of the reduction ordering.  For all orderings $0, 1, *, +, -$
are wired in the corresponding ordering option as default. The actual
ordering can be printed on the screen with {\tt (ord-show)}. The
automatically generated one is automatically sort compatible. The user
has to keep his one also sort compatible, else the inference system is
incomplete.

\PO
\oline{KNUTH-BENDIX}{The ordering as defined by D.\ Knuth and P.\ Bendix 
in their historical paper \cite{KnBe70}. The operator ordering 
is defined via ER\_OPERATOR.ORDERING. The weights are defined using 
the option ER\_KNUTH.BENDIX.WEIGHT. }
\olineeins{KNUTH-BENDIX-REVERSE}
\oline{}{Counting subterms in reverse order, i.e.\ left associative.}
\oline{POLYNOMIAL}{Polynomial ordering. The polynomials are defined with the option ER\_POLYNOMIAL.WEIGHT.}
\oline{RECURSIVE-PATH}{For the recursive path ordering the ordering of the operators
 is determined according to 
the option  ER\_OPERATOR.ORDERING. It is based on multi set orderings to compare sets of subterms.}
\olineeins{$^*$LEXICOGRAPHIC-RECURSIVE-PATH}
\oline{\quad }{Usual extension of the Recursive Path Ordering replacing the multiset ordering by a lexicographic one.}
\pagebreak

\subsubsection{ER\_OPERATOR.ORDERING}
\index{{\sc er\_operator.ordering}}
\index{operator ordering} \index{option operator ordering} \index{equality option operator ordering}

The operator ordering for all orderings.

\PO
\olineeins{A list of arbitrary valid {\sc Mkrp}-symbols.}
\oline{Default value}{($*\ -\ +\ 0\ 1$))}

\subsubsection{ER\_KNUTH.BENDIX.WEIGHT}
\index{{\sc er\_knuth.bendix.weight}}
\index{knuth bendix weight} \index{option knuth bendix weight} \index{equality option knuth bendix weight}

  An associationlist associating function and constant names to
values.

\PO

\oline{Default value}{$((+\ 1)\ (*\ 1)\ (-\ 0)\ (0\ 1)\ (1\ 1))$}

\subsubsection{ER\_POLYNOMIAL.WEIGHT}
\index{{\sc er\_polynomial.weight}}
\index{polynomial weight} \index{option polynomial weight} \index{equality option polynomial weight}

  Specification of the polynomials for the function symbols and the
values for the constants in form of an association list.
Each entry is a triple ({\em name arguments polynomial\/}), where {\em
name\/} is the name of a function, constant, or predicate symbol and args
is the list of names of the variables in the polynomials.  The
variables are associated in the given order to the subterms of terms.
{\em Polynomial\/} is a polynomial with +, *, and \^ all two place functions
and positive integers.

Syntax of polynomials:

\begin{tabbing}		
atom \quad \= ::= \quad \= \ \kill		    
pol \> ::= \> (+ mon ... mon) {\tt |} mon\\
mon \> ::= \> (* exp ... exp) {\tt |} exp\\
exp \> ::= \> (\^ exp $<$number$>$) {\tt |} $<$number$>$ {\tt |} $<$variable$>$\\
$<$number$>$ \>\> is positive integer or zero.\\
$<$variable$>$ \>\> is one of the argument variables.
\end{tabbing}

 
\PO
\olineeins{An association list associating polynomial to names.}
\oline{Default value}{%
\vspace{-1em}
\begin{tabbing}
(\=(+ (X Y) (+ X (* 2 Y)))\\
\> (* (X Y) (+ X (* X Y)))\\ 
\> ($-$ (X) (* X X))
\\ \> (0 () 2)\\ \> (1 () 2))
\end{tabbing}\hfill}


\subsubsection{ER\_NARROW.DEPTH}
\index{{\sc er\_narrow.depth}}
\index{narrow depth} \index{option narrow depth} 
\index{equality option narrow depth}\label{ernarrowdepth}


After the discovery of canonical rewrite systems attempts were made to use
them in a wider context than simple reduction, such that not only the
equality of terms can be decided but proper solutions for equations can be
computed. By this way the matching in the directed application of equations
for rewriting is replaced by complete unification. Of course this is no
reduction operation and hence the complete search space must be regarded,
which is nevertheless considerably reduced by the orientation of the
equations. For this constrained usage of equations the notion ``narrowing'' was
coined.

M.\ Fay and J.\ M.\ Hullot \cite{Fay79,Hullot80} were the first
to use narrowing techniques to construct unification 
algorithms for a  canonical theory, J.\ M.\ Hullot found further
constraints and called his procedure basic narrowing, that means 
that postions that are prefixes of the narrow position must 
not be considered for further steps.

L.\ Fribourg \cite{Fribourg84,Fribourg85a,Fribourg85b} made additional
restrictions, he used an innermost strategy, but for his method
more conditions must hold for the equational theory. 

This parameter determines the depth of the look-ahead via narrowing.

\PO

\olineeins{Integers between 0 and 10 inclusive.}
\oline{Default value}{0}   

\subsubsection{ER\_NARROW.NEXT}
\index{{\sc er\_narrow.next}}
\index{narrow next} \index{option narrow next} \index{equality option narrow next}

    Determines how the heuristic is computed that decides which
equation in the narrow-tree is inspected next:

\PO
\oline{$^*$:DEPTH}{The depth of the narrow-step.}
\oline{:EQ}{The number of symbols in the equation.}
\oline{:CLAUSE}{The number of symbols in the equation and the additional literals.}
\oline{:TAU}{The number of symbols in the substitution that instantiates the
          original equation to the actual one.}
\oline{:ALL}{The sum of :DEPTH and :EQ.}
\oline{:TRICK}{Allows to use an own function (USER:NAR=TRICK).}

\subsubsection{ER\_NARROW.TEST}
\index{{\sc er\_narrow.test}}
\index{narrow test} \index{option narrow test} \index{equality option narrow test}

    Choosing the narrowing-strategy and -tests. For a detailed
    description see \cite{Krischer90} and \cite{Richts91}.

\oline{Default value}{(:NORM :C :DELTA :SL :N)}
\oline{:NORM}{Normal-narrowing: normalization after each narrow-step.}
\oline{:C}{Commutative-narrowing.}
\oline{:N}{Normalization-test: successful if the substitution that instantiates
          the original equation to the actual one is reducible.}
\oline{:DELTA}{Delta-test: successful if the substitution of a narrow-step from one
          equation has been computed for another narrow-step before.}
\oline{:SL}{Sufficient-large-test: successful if a non-basic occurrence is reducible.}

The options ER\_NARROW.NEXT and  ER\_NARROW.TEST are actually not used.

\subsubsection{ER\_COMPILE}
\index{{\sc er\_compile}}
\index{compile} \index{option compile} \index{equality option compile}

Controls the compilation of rewrite rules into Lisp code (see
\cite{Praecklein92}).

\pagebreak
\PO
\oline{$^*$NIL}{Single non compiled rules are used for rewriting.}
\oline{1 - 1000}{Specifies the number of steps after which
                 the single rewrite rules are compiled.}
\oline{TREE-INTERPRETER}{A tree is constructed for each function symbol
                 and dynamically updated if deleting and inserting
rules.}



\subsection{Options of the Category GEN}

This area consists of several general options.

\subsubsection{GEN\_SPLITTING}
\index{{\sc gen\_splitting}}
\index{splitting} \index{option splitting} \index{general option splitting}

This option allows partitioning of the theorem into several
independent subproblems if the theorem contains conjunctions
(Reduction of a problem in several smaller subproblems). The option
causes a transformation of the theorem in DNF and then the DNF is
splitted.

\Ex

$[(\exists x \forall y Px = Py) \Leftrightarrow ((\exists x Qx)
\Leftrightarrow (\forall y Py))]
\quad\Leftrightarrow\quad
[(\exists x \forall y Qx = Qy) \Leftrightarrow ((\exists x Px)
\Leftrightarrow (\forall y Qy))]$

This is an example giben by Andrews. Direct transformation into clausal
normal form would generate 128 clauses each having 8 literals to be
inserted into the graph.  Splitting of the theorem with depth 2 produces
8 independent graphs with 8 clauses each of which has two literals, that
is, 8 relatively trivial subproblems must be solved.

\PO
\oline{NIL/N/NO}{Option is switched off.}
\oline{Integer $\ge$ 0}{Switched on. Maximal nesting depth up to which 
multiplication into DNF takes place in order to enable splitting.}
\oline{T/Y/YES}{Switched on. Multiplication in all nesting depth.}
\oline{Default Value}{0}

\subsubsection{GEN\_PRESIMPLIFICATION}
\index{{\sc gen\_presimplification} }
\index{presimplification} 
\index{option presimplification} \index{general option presimplification}
   Removing obviously true and false components and expanding
definitions.\index{definitions}\index{expanding definitions}

Definitions are formulae of the form {\tt ALL X,Y P(X Y) EQV
<}formula{\tt >} with different variables if {\tt P} does not occur in
{\tt <}formula{\tt >}. They can be expanded by replacing all
occurrences of {\tt P}s by an instance of the defining {\tt
<}formula{\tt >}.

\Ex

{\tt ALL X,Y,Z P(X Y) EQV ...} is {\em not} a definition.

{\tt ALL X,Y P(X Y) EQV ALL Z ...} is a definition.

The quantified variables must have the same sort as specified for {\tt
P}.

\Ex

{\tt ALL N,M:NAT P(N M) EQV ...} is a definition, only if {\tt P} is
defined by {\tt TYPE P(NAT NAT)}.

\pagebreak
\PO
\oline{NIL/N/NO}{Switched off.}
\oline{PARTIAL/P}{Switched on only the removal of obviously true and false components.}
\oline{$^*$T/Y/YES}{Switched on also for exanding of definitions.}

\subsubsection{GEN\_MIN.EXPRESSION.LENGTH.FOR.FILE}
\index{{\sc gen\_min.expression.length.for.file}}
\index{min expression length for file} 
\index{option min expression length for file}
\index{general option min expression length for file}
    Usage of a file to store the splitparts:

\PO
\oline{$^*$NIL/N/NO}{Store in virtual memory.}
\oline{T/Y/YES}{Store in file.}
\oline{Natural number}{If expression resulting from normalization is longer
           than this number, store in file, else in virtual memory.}

\paragraph{Remark}
This option is only useful for problems with intensive normalization
and on machines with small virtual memory.

\subsubsection{GEN\_MIN.EXPRESSION.SIZE.FOR.FILE}
\index{{\sc gen\_min.expression.size.for.file}}
\index{min expression size for file}
\index{option min expression size for file}
\index{general option min expression size for file}
    Usage of a file to store the splitparts:

\PO
\oline{$^*$NIL/N/NO}{Store in main memory.}
\oline{T/Y/YES}{Store in file.}
\oline{Natural number}{If expression resulting from normalization is larger
           than this number, store in file, else in virtual memory.}

\paragraph{Remark}
This option is only useful for problems with intensive normalization
and on machines with small virtual memory.


\subsubsection{GEN\_MANUAL.CONTROL}
\index{{\sc gen\_manual.control}}
\index{manual control} \index{option manual control} 
\index{option manual control}\label{genmanualcontrol}

Allows the user to influence the proof. Links to be used for deduction can 
be chosen interactively by the user. The potential result of active
links is labeled with an asterisk. The manual selection can be slow if
a lot of links exist and the menu facility is used to select among them.

\pagebreak
\PO
\oline{T/Y/YES}{The option is switched on.}
\oline{$^*$NIL/N/NO}{The option is switched off.}

\paragraph{Remark}
This option is very useful for debugging the input formula set
(completeness of axioms, correct sorts). Here one can enter a hand
made proof and examine whether the system is able to execute it.


\subsubsection{GEN\_MAXIMUM.STEPS}
\index{{\sc gen\_maximum steps}}
\index{maximum steps} \index{option maximum steps} \index{general option maximum steps}

Allows to limit the number of deduction steps of a proof. When reaching 
the specified number of steps {\sc Mkrp} stops with the message 
``RESULT: FAILURE.ABORTED.MAXSTEPS''.

\PO
\oline{$^*$NIL}{Infinite.}
\oline{Positive integer}{Maximum number of deduction steps of a proof.}

\paragraph{Remark}
This option is important especially if the prover runs without
supervision of a user. If for example the machine runs out of swap space
and no refutation has been 
found so far, the prover stops ``abnormal'', that is, it is not possible to create a 
listing of the deduction steps made so far. If one can approximate the 
maximal number of deduction steps, GEN\_MAXIMUM.STEPS can be set to this 
number. After  reaching the specified step the deduction process stops  
and it is now possible to create a listing to evaluate the successless work
of {\sc Mkrp}. 

\subsubsection{GEN\_GRAPH.SAVING}
\index{{\sc gen\_graph.saving}}
\index{graph saving} \index{option graph saving} \index{general option graph saving}

Enables the user to store the actual state of the graph after a certain 
number of deduction-steps in order to have the possibility to restart the 
deduction process at this state. It is useful to save the graph a few steps 
before GEN\_MAXIMUM.STEPS is reached, so that the graph can be continued, 
if this seems promising. The save file can be very big, that is, up to some
MByte.

\PO
\oline{$^*$NIL}{No effect.}
\oline{Positive integer}{Number of deduction-steps between two savings of the                                     
graph.}

\paragraph{Remark} The option GEN\_SAVE.FILE allows specifying a file-name for the 
graph  stored by the user.

\subsubsection{GEN\_SAVE.FILE}
\index{{\sc gen\_save.file}}
\index{save file} \index{option save file} \index{general option save file}

Allows to specify a file name on which the graph(s) (see 
GEN:GRAPH.SAVING) will be stored. 

\pagebreak
\PO
\oline{$<$File name$>$}{The system saves the graph on this file.}
\oline{$^*$NIL/N/NO}{A default name for the saved graph will be used.}


\subsubsection{GEN\_LISP.GARBAGE.COLLECTION}
\index{{\sc gen\_lisp.garbage.collection}}
\index{lisp garbage collection} \index{option lisp garbage collection} \index{general option lisp garbage collection}
    Switch on the lisp garbage collector before address space becomes low.

\PO
    \oline{T/Y/YES}{Switched on.}
    \oline{$^*$NIL/N/NO}{Switched off.}

\subsubsection{GEN\_COMMON.LISP}
\index{{\sc gen\_common.lisp}}
\index{common lisp} \index{option common lisp} \index{general option common lisp}
    Says whether to use pure Common Lisp.

\PO
    \oline{$^*$T/Y/YES}{Use pure Common Lisp.}
    \oline{NIL/N/NO}{Use Symbolics or X-window features (editor options).}

\subsubsection{GEN\_OTHER.PROVER}

Using another theorem prover.

\oline{mkrp}{A clause graph is printed on the graph file in a form readable for the REFUTE subsystem.}
\oline{C}{A compiling version of the equality handling is used.
Sorts and terminator don't work for
this option.  Strategy corresponding to \cite{BaGa90},
Ordering is the same as for HC strategy. This procedure is implemented
in C and must be handled in a different way than {\sc Mkrp}. It is
only used as input module to the C-program. The graph file is a compilable MAIN program for C.}
\oline{ER}{The graph file is a compilable LISP program.}





On a Symbolics or if an X-window capable system and terminal is used,
the values of the options can be chosen using the mouse and a  
window oriented menu system.
