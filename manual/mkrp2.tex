\chapter{The MKRP Operating System}
\label{TheMKRPOperatingSystem}

\section{Installation Guide for MKRP}
\label{InstallationGuideforMKRP}

You can select whether to have a German or English user interface for 
{\sc Markgraf Karl}.
\index{language}\index{dialogue language}\index{English}\index{German}
The default is English. To get the German one, you must put the
atom \verb!:mkrp.deutsch! into the list stored in 
the Common Lisp variable \verb!*features*! before compiling the system.

To use theory unification you need the {\sc Hades}-system, for the
c-prover (option GEN\_C.PROVER) the {\sc Erm}-system must be available.

\subsection{Installation on Symbolics Lisp Machines}
\label{InstallationonSymbolicsLispMachines} 
\begin{enumerate}
\item    Place system and translation file in your site directory and update the pathnames therein.
\item    Use the command ``:compile system MKRP'' to compile the system.
\item    Use the command ``:load system MKRP'' to load it.
\end{enumerate}
\index{installation}

\subsection{Installation on Other Machines}
\label{InstallationonotherMachines}
\begin{enumerate}
\item  	Select ``mkrp'' as working directory, there should be the
	subdirectories ``prog'', ``sys'', and ``examples''. 
You can delete examples if you don't have enough room.
\item  	Call your Lisp.
\item  	Call {\tt (load "sys/cl-boot.lisp")}
\item  	Call {\tt (in-package "MKRP")}
\item  	Call {\tt (mkrp-boot)}, a menu appears, select 
``Newly Compile all Files''.
There will be some undefined functions in the package ``HADES'',
you need not worry about.
\item 	After the compilation has finished you can do the same procedure with
selecting ``Load Compiled Files'' to use {\sc Mkrp}.
Alternatively you can save an image calling {\tt (mkrp-dumper)} which 
should generate a file {\tt "mkrp/mkrp"}.
\end{enumerate}

\subsection{Starting MKRP}

\index{start MKRP}
After loading the system in the same way as described in
section \ref{InstallationonotherMachines} you have to call
{\tt (mem-initialize 10000)} and {\tt (mkrp)} (see also 
\ref{UsingmkrpWithoutUserDialogue}). This is not necessary if you used
the dumper.


\section{Architecture of the System from a User's View}
\label{ArchitectureoftheSystemfromaUser'sView}

The system consists of the following four subsystems: \index{subsystems}

		EDIT
		-- CONSTRUCT
		-- REFUTE
		-- PROTOCOL

\subsection{Edit}

\index{edit subsystem}
EDIT provides a set of functions to create and manipulate sets of 
axioms and theorems, enables the user to create 
{\em axiom\/}- and {\em theorem\/}-files, \index{theorem file}
\index{axiom file} \index{formulae file}
consisting of formulae of the input-language PLL (Predicate Logic
Language), which is discussed in chapter \ref{TheInputLanguage}. The
editor checks the syntactic and semantic correctness of the input
formulae. These formulae are transformed into a formal problem
description being stored in a so-called
\index{problem file}
{\em problem\/} file.

\subsection{Construct}

\index{construct subsystem}
Given a set of axioms and theorems (created by EDIT), CONSTRUCT
generates the corresponding initial connection graph(s).  The problem
file is the input for the CONSTRUCT-subsystem, which generates the
graph(s) according to the reduction rules for initial graphs as
specified by the adjustment of the options. The output of this
subsystem is a {\em graph\/} file.
\index{graph file} Raw data for the 
protocol are written to a {\em code\/} file. \label{codefile} 
\index{code file}

\subsection{Refute}

\index{refute subsystem}
With the result of CONSTRUCT as input, REFUTE tries to detect a
refutation by the application of resolution and paramodulation steps
to the connection graph. It tries to prove the theorem by resolving
clauses in the connection graph. The user can control the deduction
process by specifying various options (to be discussed in more detail
in chapter \ref{SettingtheMKRPParameters}).  REFUTE also writes raw
data to a code file. In order to get the protocol raw data of
CONSTRUCT and REFUTE on the same code file use CR (CONSTRUCT.REFUTE,
see below), else no useful protocol can be printed.

\subsection{Protocol}

\index{protocol subsystem}
PROTOCOL produces a listing of the input and the proof steps
performed. The PROTOCOL-subsystem is controlled by the PR-options (see
chapter \ref{TheOutputFacilities}).

\section {Explanation of the Operating System Commands}
\label {ExplanationoftheOperatingSystemCommands}

At the toplevel of the {\sc Mkrp}-system, the operating system 
commands of figure \ref{MKRPInterfaceCommands} are available.
\begin{figure}[ht]
\caption{{\sc Mkrp} Interface Commands}
\label{MKRPInterfaceCommands}
\begin{center}
\fbox{\parbox{15cm}{%
\begin{tabbing}
\hspace{1cm}\=\hspace{2cm}\=\hspace{2cm}\=\hspace{4cm}\=\hspace{3cm}\=\kill
   V   \>  H[ELP] \>  EX[IT]  \>  O[PTIONS]        \>  D[EFINE.]E[XAMPLE.NAME]\\
   HC  \>  L[ISP] \>  LO[GOFF]\>  V[IEW.]P[ROTOCOL]\>  D[EFINE.]D[IRECTORY]   \\
   H[ARDCOPY.]P[ROTOCOL]\\
\hspace{2cm}\=\hspace{2cm}\=\hspace{2cm}\=\hspace{2cm}\=\hspace{2cm}\=\hspace{2cm}\=\kill
 \>\>\>                               EP\\
 \>\>\>                               FP\\
 \>\>                         ER      \>\>      CP\\
 \>\>                         FR\\
 \>                   EC      \>\>      CR      \>\>      RP\\
 \>                   FC\\
     E[DIT]    \>\>    C[ONSTRUCT]  \>\>      R[EFUTE]   \>\> P[ROTOCOL]\\
      F[ORMULA]
\end{tabbing}}}
\end{center}
\end{figure}

\parbox[t]{4.9cm}{D[EFINE.]D[IRECTORY] 	

\index{define directory} \index{command define directory}
\quad $<$Directory$>$}
\hfill\parbox[t]{11cm}{
	Defines the directories where the files 
will be written on and read from unless specified otherwise. It is only useful together with DE.}

\parbox[t]{4.9cm}{D[EFINE.]E[XAMPLE.NAME]

\index{define example name} \index{command define example name}
\quad $<$Example-Name$>$}
\hfill\parbox[t]{11cm}{Files generated during the session will be put into the subdirectory $<$Example-Name$>$ of the directory defined with DD.

\begin{tabbing}
Example:\quad\= DD  \= /user2/mkrp/examples/\\
\>                DE  \> steamroller
\end{tabbing}

All files are generated in /user2/\-mkrp/\-examples/\-steam\-roller/} 

\index{video control} \index{command video control}
\parbox[t]{3cm}{V  T$|$Y$|$YES

V  NIL$|$N$|$NO}\hfill\parbox[t]{11cm}{Turns the manual terminal 
control on or off. This doesn't work for all machines.}

\parbox[t]{3cm}{H[ELP] }\hfill\parbox[t]{11cm}{Prints a list of all 
available operating system commands on the screen.
\index{help} \index{command help}}

\parbox[t]{3cm}{H[ELP] $<$Com$>$}\hfill\parbox[t]{11cm}{Explains the 
command $<$Com$>$.}

\parbox[t]{3cm}{EX[IT]}\hfill\parbox[t]{11cm}{Terminates the {\sc
Mkrp}-session and returns to Lisp. \index{exit} \index{command exit}}

O[PTIONS]\hfill\parbox[t]{11cm}{Calls the option module for setting
the parameters of proof-control. The actual options can be stored in
an options file.  ``OK'' returns to where you came from. The option
module is described in detail in chapter \ref{SettingtheMKRPParameters}.
\index{options} \index{command options} \index{options file}}

\parbox[t]{4cm}{H[ARD]C[COPY]   $<$File$>$

\index{hardcopy} \index{command hardcopy} \index{print hardcopy}
HC     T

HC    N }\hfill\parbox[t]{11cm}{Input and output of the session 
is printed on $<$File$>$, with T output is directed to the printer, with 
NIL hardcopy is switched off and $<$File$>$ is closed.}

LO[GOFF]\hfill\parbox[t]{11cm}{Terminates this session.
\index{logoff} \index{command logoff}}

E[DIT] [$<$Problem File$>$]\hfill\parbox[t]{11cm}{Creates a problem
description. To that end the formula editor is called
\index{edit} \index{command edit}
twice, 	first for the axiom formulae, then for the theorem
formulae.  The problem description is written on $<$Problem File$>$,
if given, otherwise a default name is used.}

\parbox[t]{4cm}{F[ORMULA]

\index{formula} \index{command formula}
\quad $<$Axiom  File$>$  

\quad$<$Theorem File$>$ 

\quad[$<$Problem File$>$] }
\hfill\parbox[t]{11cm}{
Creates a problem description from formula files.
 
$<$Axiom File$>$ and $<$Theorem File$>$ contain the compiled (written
with the WRITE editor command) axiom and theorem formulae or formulae
acceptable to the GET editor command. Their default names are {\tt
"ax.lisp"} and {\tt "th.lisp"}. The problem description is written to
$<$Problem File$>$, if given, otherwise the default name {\tt
"problem.lisp"} is used. Using this command requires $<$Axiom File$>$
and $<$Theorem File$>$ to be compatible, i.e.\ they must be created
during the same editor session, if written with WRITE.  Another
possibility is that $<$Axiom File$>$ and $<$Theorem File$>$ are both
in a format acceptable to the GET command of the formula editor (see
chapter \ref{TheMKRPEditor}).}


\parbox[t]{4cm}{C[ONSTRUCT]

\index{construct} \index{command construct}
\quad [$<$Problem File$>$

\quad [$<$Graph File$>$

\quad [$<$Code File$>$ 	

\quad [$<$Comment$>$]]]]}
\hfill\parbox[t]{11cm}{Creates a set of initial graphs from a 
problem description.  $<$Problem File$>$ contains the problem
description. If not specified, the last one created is used, if such a
file exists in the directory defined with DD and DE.
$<$Graph File$>$ is the file the initial graph(s) will be
written on, and defaults to {\tt "graph.lisp"} if NIL is used.
$<$Code File$>$ is the file, where the raw data for the protocol
will be written to. If NIL is used,
the default file name {\tt "code.lisp"} is used.
$<$Comment$>$ is inserted into the proof protocol. It must be a list
or string, each element is printed in a separate line.}

\parbox[t]{4cm}{R[EFUTE]

\index{refute} \index{command refute}
\quad[$<$Graph File$>$

\quad[$<$Number$>$

\quad[$<$Code File$>$]]]}
\hfill\parbox[t]{11cm}{
Refutes a set of initial graphs and creates raw data for the
protocol. $<$Graph File$>$ is the file containing the initial
graph(s).  If NIL is given, the last one created is used, if such a file exists
in the directory defined with DD and DE.  If a $<$Number$>$ say $n$,
is given, only the $n$th graph is refuted, otherwise (i.e.
$<$number$>$ = NIL) all graphs on the file.  		$<$Code
File$>$ is the file, where the raw data for the protocol will
be written to. 
If C and R commands
are executed separately the protocol will contain only information
about construction or refutation. To get a complete protocol you must
use the combined command CR to generate a code file.}

\parbox[t]{4cm}{P[ROTOCOL]\index{protocol}\index{command
protocol}

\quad[$<$Code File$>$

\quad[$<$List File$>$]]}
\hfill\parbox[t]{11cm}{
Creates a proof protocol from raw data.
$<$Code File$>$ is the file containing the raw data.
$<$List File$>$ is the file, where the processed data for protocol
will be written to in a readable format.
The default is {\tt "list.text"}.}

The commands Edit, Formula, Construct, Refute, and Protocol can be 
combined in various different ways. 
This causes several commands being executed after one another. 
Possible combinations are depicted in figure \ref{MKRPControlCommands}.
\begin{figure}[ht]
\caption{MKRP Control Commands}
\label{MKRPControlCommands}
\begin{center}
\fbox{\parbox{10cm}{%
%\begin{center}
\begin{tabbing} 
\hspace{2cm}\=\hspace{2cm}\=\hspace{4cm}\=\hspace{3cm}\=\kill
\hspace{3cm}\=\hspace{2cm}\=\hspace{2cm}\=\hspace{2cm}\=\kill
 \>\>                              EP\\
 \>\>                               FP\\
 \>                         ER      \>\>      CP\\
 \>                         FR\\
                   EC      \>\>      CR      \>\>      RP\\
                    FC
\end{tabbing}
%\end{center}
}}
\end{center}
\end{figure}
\index{command EP}
\index{command FP}
\index{command ER}
\index{command CP}
\index{command FR}
\index{command EC}
\index{command CR}
\index{command RP}
\index{command FC}

CP $<$Problem File$>$ $<$Graph File$>$ $<$Code File$>$
                               $<$Comment$>$ $<$List File$>$ for example, 
causes first CONSTRUCT to create a $<$Code File$>$ starting 
with $<$Problem File$>$. Then REFUTE tries to refute the graph and writes also 
protocol raw data to the $<$Code File$>$  and at last
PROTOCOL creates a $<$List File$>$. Any command combining CONSTRUCT and REFUTE
produces an empty graph file.
\section{Using MKRP Without User Dialogue}
\label{UsingmkrpWithoutUserDialogue}

The system can be used without entering the dialogue system by calling 
{\tt (mkrp} $<$Directory$>$ $<$Example-Name$>$ $<$Hardcopy$>${\tt )}, where 
$<$Directory$>$ and $<$Example-Name$>$ \index{start MKRP} are such that they 
could serve as arguments to the 
operating system commands DD and DE, respectively. They specify the directory where the 
system searches for input files and places output files. The input is done either 
from axiom and theorem files or from a problem file according as they exist.
If both exist the newer alternative is selected.

The system performs an FP or CP command. If $<$Hardcopy$>$ is true the output of the 
protocol is sent to the default printer. All terminal traces are
directed to a file named {\tt "batch.text"}.
