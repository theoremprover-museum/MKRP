\chapter{The Input Language}
\label{TheInputLanguage}


An overview on PREDICATE LOGIC LANGUAGE (PLL), a formal language in
which sorted \index{predicate logic language} \index{PLL} first-order
predicate logic formulae can be formulated, is given.  An exact
description is given in \cite{Walther82}. Axioms and theorems, which
are given to {\sc Markgraf Karl}, are written in PLL. The language
constructs of PLL which reflect the special facilities of this system
are

\begin{itemize}
\item 	an inference mechanism based on an order-sorted calculus, 
\index{order-sorted} \index{sorted}
\item	the incorporation of special axioms into the inference mechanism, and
\item	the control of the inference mechanism using special derivation 
	strategies. 
\end{itemize}

\section{Basic Concepts}
\label{BasicConcepts}

In PLL the usual junctors, denoted by {\tt OR}, {\tt AND}, {\tt IMPL}, 
{\tt EQV}, and {\tt NOT}, the\index{junctors}\index{quantors}
universal quantifier {\tt ALL} and the existential quantifier {\tt EX} are present. 
Junctors and quantifiers have the following priorities when used in a 
formula without parentheses:

{\tt NOT} $>$ {\tt AND} $>$ {\tt OR} $>$ {\tt IMPL} $>$ {\tt EQV} $>$ 
{\tt ALL}, {\tt EX}


In a formula without parentheses, the rightmost junctor has precedence 
over all junctors of the same priority to its left.

\Ex

{\tt NOT A OR B AND C} is equivalent to {\tt (NOT A) OR (B AND C)} and {\tt A IMPL B IMPL 
C} is equivalent to {\tt A IMPL (B IMPL C)}. 

In PLL the sign {\tt =} denotes the equality symbol, that is, we use a first-order 
predicate calculus with equality. The equality sign is written infix in contrast to all other 
predicates. As an example for using PLL, we 
axiomatize a group:

\begin{verbatim}
* AXIOMATIZATION OF A GROUP WITH EQUALITY
* F IS A GROUP OPERATOR, I THE INVERSE, AND 1 IS THE IDENTITY ELEMENT 
ALL X          F(I(X) X) = 1
ALL X,Y,Z      F(X F(Y Z)) = F(F(X Y) Z)
ALL X          F(1 X) = X 
\end{verbatim}
A theorem given to the {\sc Mkrp}-system could be for instance: 

\begin{verbatim}
* IDEMPOTENCY IMPLIES COMMUTATIVITY
ALL X F(X X) = 1 IMPL (ALL X,Y F(X Y) = F(Y X))
\end{verbatim}

The lines starting with asterisks are PLL-comments. We give another 
axiomatization of a group:

\pagebreak[3]
\begin{verbatim}
* AXIOMATIZATION OF A GROUP WITHOUT EQUALITY
* P(X Y Z) DENOTES F(X Y) = Z WHERE F IS THE 
* GROUP OPERATOR, I IS THE INVERSE, AND 1 IS THE LEFTIDENTITY.
ALL X             P(I(X) X 1) 
ALL X,Y,Z,U,V,W   P(X Y U) AND P(Y Z V) IMPL (P(X V W) EQV P(U Z W))
ALL X             P(1 X X) 
\end{verbatim}

Now a theorem could be for instance:
\begin{verbatim}
* LEFTIDENTITY IS RIGHTIDENTITY
ALL X P(X 1 X)
\end{verbatim}

\section{The Order-Sorted Calculus}
\label{TheMany-SortedCalculus}

Assume we have a set of symbols ordered by the subsort ordering, a
partial ordering relation which is reflexive, antisymmetric, and
transitive. The subsort ordering has a predefined upper bound named top
sort ({\tt Any}). Variable, constant, and function symbols are
associated with a certain sort symbol.  The sort of a variable or
constant symbol is its rangesort and the sort of a term which is
different from a variable or constant symbol is determined by the
rangesort of its outermost function symbol.

All argument positions of a function or predicate symbol are
associated with certain sort symbols, called the domainsorts. In the
construction of the formulae of the order-sorted calculus, only those
terms may fill an argument position of a function or predicate symbol,
whose sorts are subsorts of the domainsorts given for the argument
position of the respective function or predicate symbol.

Besides the increase of readability of axiomatizations, the usage of
the information given by the range- and domainsorts and by the subsort
ordering prevents the inference mechanism of the theorem prover to do
useless derivations. The theoretical foundation of the order-sorted
calculus implemented in the {\sc Markgraf Karl Refutation Procedure}
can be found in \cite{Walther84}. 

As an example for an application of a order-sorted calculus we axiomatize 
sets of letters and digits and some basic operations for these sets.
 
\Ex
\begin{verbatim}
* LETTER AND DIGIT ARE SUBSORTS OF SORT SIGN
SORT SIGN: ANY
SORT LETTER,DIGIT: SIGN
\end{verbatim}

\begin{verbatim}
* DEFINITION OF THE SORTS LETTER AND DIGIT; I.E.
* A, B, ... , Z ARE CONSTANTS OF SORT LETTER AND  
* 0, 1, ... , 9 ARE CONSTANTS OF SORT DIGIT
TYPE A,B,C,D,E,F,G,H,I,J,K,L,M,N,O,P,Q,R,S,T,U,V,W,X,Y,Z: LETTER
TYPE 9,8,7,6,5,4,3,2,1,0: DIGIT
\end{verbatim}

\begin{verbatim}
* DEFINITION OF THE EMPTY SET AND SET-MEMBERSHIP,
* I.E. EMPTY IS A CONSTANT OF SORT SET AND MEMBER
* IS A BINARY PREDICATE DEFINED ON (SIGN SET)
SORT SET: ANY
TYPE EMPTY: SET
TYPE MEMBER(SIGN SET)
ALL X:SIGN NOT MEMBER(X EMPTY)
ALL U,V:SET U = V EQV (ALL X:SIGN MEMBER(X U) EQV MEMBER(X V))
\end{verbatim}

\begin{verbatim}
* DEFINITION OF SINGLETONS, I.E.
* SINGLETON IS A FUNCTION MAPPING SIGN TO SET
TYPE SINGLETON(SIGN):SET
ALL X,U:SIGN MEMBER(U SINGLETON(X)) IMPL U=X
\end{verbatim}

\begin{verbatim}
* DEFINITION OF UNION, I.E.
* UNION IS A FUNCTION MAPPING (SET X SET) TO SET
TYPE UNION(SET SET):SET
ALL X:SIGN ALL U,V:SET (MEMBER(X U) OR MEMBER(X V)) EQV MEMBER(X UNION(U V))
\end{verbatim}
Theorems to be proved by the ATP system could be for instance:
\begin{verbatim}
* UNION IS IDEMPOTENT AND EMPTY IS AN IDENTITY ELEMENT
ALL X:SET UNION(X X) = X AND UNION(EMPTY X) = X
\end{verbatim}

\begin{verbatim}
* SINGLETON IS INJECTIVE
ALL X,Y:SIGN SINGLETON(X) = SINGLETON(Y)  IMPL X = Y
\end{verbatim}

\begin{verbatim}
* EACH LETTER IS A SIGN
ALL Y:SET (EX U:LETTER MEMBER(U Y)) IMPL (EX X:SIGN MEMBER(X Y))
\end{verbatim}

\section{Attributes of Functions and Predicates}
\label{AttributesofFunctionsandPredicates}

Attributes are abbreviations for their defining axioms, i.e.\ first-order 
axioms which axiomatize certain properties of functions or predicates. 

The effect in stating a certain attribute of a function or predicate using an 
attribute declaration is formally the same as giving the defining axiom to 
the ATP. At the moment the following properties can be declared. 

\index{reflexive predicate} \index{irreflexive predicate} \index{symmetric 
predicate} \index{associative function} \index{commutative function}
\index{theory unification}
\begin{tabular}{lll}
Attribute Declaration &	Defining Axioms\\
{\tt REFLEXIVE(P)}&		{\tt ALL X}&	{\tt P(X X)}\\
{\tt IRREFLEXIVE(P)}&		{\tt ALL X}&	{\tt NOT P(X X)}\\
{\tt SYMMETRIC(P)}&		{\tt ALL X,Y}&	{\tt P(X Y) IMPL P(Y X)}\\
{\tt ASSOCIATIVE(F)}&		{\tt ALL X,Y,Z}&   {\tt F(X F(Y Z)) = F(F(X Y) Z)}\\
{\tt AC(F)}&		{\tt ALL X,Y,Z}&	{\tt F(X F(Y Z)) = F(F(X Y) Z)}\\
	&		{\tt ALL X,Y}&   	{\tt F(X Y) = F(Y X)}\\
{\tt AC1(F F1)}&		{\tt ALL X,Y,Z}&	{\tt F(X F(Y Z)) = F(F(X Y) Z)}\\
	&		{\tt ALL X,Y}&   {\tt F(X Y) = F(Y X)}\\
	&		{\tt ALL X}&   	 {\tt F(X F1) = X}\\
{\tt AG(F F- F1)}&	{\tt ALL X,Y,Z}& {\tt F(X F(Y Z)) = F(F(X Y) Z)}\\
	&		{\tt ALL X,Y}&   {\tt F(X Y) = F(Y X)}\\
	&		{\tt ALL X}&   	 {\tt F(X F1) = X}\\
	&		{\tt ALL X}&   	 {\tt F(X (F- X)) = F1}
\end{tabular}

The theory unification facility is not realized inside {\sc Mkrp} itself, the 
HADES-system must be loaded to use {\tt ASSOCIATIVE}, {\tt AC}, {\tt
AC1}, and {\tt AG}.
\label{Attributes}


\Ex

The associativity of the group operator F 
could be stated by: \verb!ASSOCIATIVE(F)!.

\paragraph{Remark}
It is not advisable to use attribute declarations for function symbols 
because they slow down the system remarkably.


\section{The Syntax of PLL}
\label{TheSyntaxofPLL}

The syntax of PLL is defined by the following context free grammar:
\index{grammar for PLL}

{\em Terminal Alphabet\/}

{\tt SORT TYPE ANY TRUE FALSE = =:}{\tt \ }{\tt := :=:

ASSOCIATIVE AC AC1 AG REFLEXIVE IRREFLEXIVE SYMMETRIC

ALL   EX   EQV   :EQV   EQV:   :EQV:   IMPL   :IMPL   IMPL:  :IMPL:

OR   :OR   OR:   :OR:   AND   :AND   AND:   :AND:   NOT

A  B  C  D  E  F  G  H  I  J  K  L  M  N  O  P  Q  R  S  T  U  V  W  X  Y  Z

0  1  2  3  4  5  6  7  8  9

: ; * ! \# \$ \& - @ + , / . ( ) ? " + $>$ $<$ [ ]}

{\em Non-terminal Alphabet\/}

Every string enclosed in angle brackets, e.g.\ $<$term$>$.
$<$$>$ denotes the empty word.

{\em Start Symbol\/}:\quad
$<$expression$>$

\def \prule#1#2{\parbox[t]{4cm}{#1}\hfill $\rightarrow$\hfill\parbox[t]{11cm}{#2}}

\pagebreak
{\em Production Rules\/}

\prule{$<$expression$>$}%	
{$<$$>$   {\tt |}   $<$comment$>$   $<$expression$>$ {\tt |} \\
$<$sort  de\-cla\-ra\-tion$>$   $<$ex\-pres\-sion$>$   {\tt |}\\
$<$type  declaration$>$   $<$expression$>$   {\tt |}\\
   $<$attribute  declaration$>$   
$<$expression$>$   {\tt |}\\
$<$quantification$>$   $<$expression$>$  }

\prule{$<$comment$>$}%
{{\tt *} `any sequence of symbols'}

\prule{$<$sort declaration$>$}%
{{\tt SORT }$<$sort list$>$ {\tt :} $<$sort list$>$}

\prule{$<$type declaration$>$}%
{{\tt TYPE } $<$constant  list$>$ {\tt :} $<$sort symbol$>$ {\tt |}\\
 {\tt TYPE } $<$function symbol$>$  ($<$sort symbols$>$) {\tt :} $<$sort symbol$>$ {\tt |}\\
{\tt TYPE } $<$predicate symbol$>$  ($<$sort symbols$>$)}

\prule{$<$attribute declaration$>$}%
{{\tt ASSOCIATIVE} ($<$function symbol$>$) {\tt |}\\
{\tt AC} ($<$function symbol$>$) {\tt |}\\
{\tt AC1} ($<$function symbol$>$ $<$constant symbol$>$) {\tt |}\\
{\tt AG} ($<$function symbol$>$ $<$function symbol$>$
$<$constant symbol$>$) {\tt |}\\
{\tt REFLEXIVE} ($<$predicate symbol$>$)  {\tt |}\\
{\tt IRREFLEXIVE} ($<$predicate symbol$>$) {\tt |}\\
{\tt SYMMETRIC} ($<$predicate symbol$>$)}

\prule{$<$quantification$>$}%
{$<$equivalence$>$ {\tt |}\\
{\tt ALL } $<$variable declaration$>$ $<$quantification$>$ {\tt |}\\
{\tt EX } $<$variable declaration$>$ $<$quantification$>$}

\prule{$<$equivalence$>$}%
{$<$implication$>$ {\tt |}
$<$implication$>$ $<$eqv$>$ $<$equivalence$>$}

\prule{$<$implication$>$}%
{$<$disjunction$>$  {\tt |}
$<$disjunction$>$  $<$impl$>$ $<$equivalence$>$} 

\prule{$<$disjunction$>$}%
{$<$conjunction$>$  {\tt |}
$<$conjunction$>$ $<$or$>$ $<$disjunction$>$}

\prule{$<$conjunction$>$}%
{$<$negation$>$  {\tt |}
$<$negation$>$ $<$and$>$ $<$conjunction$>$}

\prule{$<$negation$>$}%
{$<$atomic formula$>$  {\tt |}
{\tt NOT} $<$atomic formula$>$}

\prule{$<$atomic formula$>$}%
{ ($<$quantification$>$) {\tt |} $<$atom$>$}

\prule{$<$eqv$>$}%
{{\tt EQV  {\tt |}  :EQV  {\tt |}  EQV:  {\tt |}  :EQV: }}

\prule{$<$impl$>$}%
{{\tt IMPL  {\tt |}  :IMPL  {\tt |}  IMPL:  {\tt |}  :IMPL: }}

\prule{$<$or$>$}%
{{\tt OR  {\tt |}  :OR  {\tt |}  OR:  {\tt |}  :OR: }}

\prule{$<$and$>$}%
{{\tt AND  {\tt |}  :AND  {\tt |}  AND:  {\tt |}  :AND: }}

\prule{$<$atom$>$}%
{$<$predicate symbol$>$  {\tt |}\\
$<$predicate symbol$>$ ($<$terms$>$)  {\tt |}\\
$<$term$>$ $<$equality symbol$>$ $<$term$>$}

\prule{$<$variable declaration$>$}%
{$<$variable list$>$  $<$variable sort$>$}

\prule{$<$variable sort$>$}%
{$<>$  {\tt |}  :  $<$sort symbol$>$}

\prule{$<$term$>$}%
{$<$constant symbol$>$  {\tt |}\\  $<$variable symbol$>$  {\tt |}\\ $<$function symbol$>$ 
($<$terms$>$)}

\prule{$<$terms$>$}%
{$<$term$>$  {\tt |}  $<$term$>$  $<$terms$>$}

\prule{$<$sort list$>$}%
{$<$sort symbol$>$  {\tt |} $<$sort symbol$>$,  $<$sort list$>$}

\prule{$<$sort symbols$>$}%
{$<$sort symbol$>$  {\tt |} $<$sort symbol$>$  $<$sort symbols$>$}

\prule{$<$constant list$>$}%
{$<$constant symbol$>$  {\tt |} $<$constant symbol$>$, $<$constant list$>$}

\prule{$<$variable list$>$}%
{$<$variable symbol$>$  {\tt |}  $<$variable symbol$>$, $<$variable list$>$}

\prule{$<$sort symbol$>$}%
{{\tt ANY} {\tt |}  $<$identifier$>$}

\prule{$<$constant symbol$>$}%
{$<$identifier$>$ {\tt |} $<$number$>$  {\tt |}  $<$name$>$}

\prule{$<$function symbol$>$}%
{$<$identifier$>$  {\tt |}  $<$name$>$}

\prule{$<$predicate symbol$>$}%
{{\tt TRUE}  {\tt |}  {\tt FALSE} {\tt |}$<$identifier$>$  {\tt |}  $<$name$>$}

\prule{$<$equality symbol$>$}%
{{\tt =  {\tt |}  :=  {\tt |}  =: {\tt |}  :=: }}

\prule{$<$variable symbol$>$}%
{$<$identifier$>$}

\prule{$<$identifier$>$}%
{$<$letter$>$  {\tt |}\\ $<$identifier$>$$<$letter$>$  {\tt |}\\
$<$identifier$>$$<$digit$>$  {\tt |}  \\
$<$iden\-ti\-fier$>$$<$spe\-cial sign$>$} 

\prule{$<$number$>$}%
{$<$digit$>$  {\tt |}  $<$number$>$ $<$digit$>$}

\prule{$<$name$>$}%
{$<$special sign$>$  {\tt |}\\ $<$name$>$$<$letter$>$ {\tt |} \\
$<$name$>$$<$digit$>$ {\tt |}\\ $<$na\-me$>$$<$spe\-cial sign$>$}

\prule{$<$letter$>$}%
{{\tt  A} {\tt  |} {\tt  B} {\tt  |} {\tt  C} {\tt |} {\tt  D} {\tt  |} {\tt  E}
{\tt  |} {\tt  F} {\tt  |} {\tt  G}
{\tt  |} {\tt  H} {\tt   |} {\tt  I} {\tt   |} {\tt  J} {\tt   |} {\tt  K} {\tt  |}
{\tt  L} {\tt  |} {\tt  M} {\tt  |} {\tt  N} {\tt  |} {\tt O} {\tt  |}
{\tt P} {\tt |} {\tt  Q} {\tt  |} {\tt  R} {\tt  |} {\tt  S} {\tt  
|} {\tt  T} {\tt  |} {\tt  U} {\tt  |} {\tt  V} {\tt  |} {\tt  W} {\tt  |} {\tt X} {\tt  |} {\tt  Y} {\tt  |} {\tt  Z }  {\tt  |} {\tt  [} {\tt  |} {\tt  ]}}

\prule{$<$digit$>$}%  
{{\tt 0 | 1 | 2 |  3 | 4 | 5 | 6 | 7 | 8 | 9 }}

\prule{$<$special sign$>$}%
{{\tt !} {\tt  |} {\tt \#} {\tt  |} {\tt  \$} {\tt  |} {\tt  \&} {\tt  |}
{\tt  *} {\tt  |} {\tt  -} {\tt  |} {\tt  +} {\tt  |} {\tt  ;} {\tt  |} 
{\tt  ?} {\tt  |} {\tt  /} {\tt  |} {\tt  .} {\tt  |} {\tt  @} {\tt  |} 
{\tt  "} {\tt  |} {\tt  +} {\tt  |} {\tt  =} {\tt  |} {\tt  $>$} {\tt  |} 
{\tt  $<$}}


The keywords {\tt SORT}, {\tt TYPE}, {\tt ANY} etc.\ are members of
the terminal alphabet and hence never accepted as $<$identifier$>$.

The {\sc Mkrp} input routine makes no difference between upper and
lower case letters, it converts all lower case letters to upper case.
 

\section{Semantic Constraints for PLL} 
\label{SemanticConstraintsforPLL}

In the sequel we state the semantic constraints (i.e.\ the context
dependent language features) for PLL. The strings in angle brackets,
e.g.\ $<$term$>$, refer to the production rules of the PLL-grammar as
defined in section \ref{TheSyntaxofPLL}.

\subsection{Sort Symbols}
\label{SortSymbols} 

Sort symbols are introduced with their first usage in 
\begin{itemize}
\item a $<$sort declaration$>$, e.g. {\tt SORT LETTER,DIGIT: SIGN,ALPHABET}
\item a $<$type declaration$>$, e.g. {\tt TYPE A,B: BOOL } 
\item a $<$variable declaration$>$, e.g. {\tt ALL Z:INT EX N:NAT ABS(Z) = N}
\end{itemize}

The direct subsort relation imposed on the set of sort symbols is a partial, 
irreflexive, and non-transitive relation such that the predefined sort 
symbol {\tt ANY} is no direct subsort of each sort symbol and each sort symbol 
different from {\tt ANY} is a direct subsort of at least one other sort symbol. 

The subsort order imposed on the set of sort symbols is the reflexive and 
transitive closure of the direct subsort relation. 

The subsort symbols left of the colon in a $<$sort declaration$>$ are direct 
subsorts of each sort symbol to the right of the colon in the $<$sort 
declaration$>$. 

The sort symbols right of the colon in a $<$sort declaration$>$ are
direct subsorts of {\tt ANY}, provided these sort symbols are
introduced by this $<$sort declaration$>$ and not by a previous one.

\Ex

For the $<$sort declaration$>$ given above {\tt LETTER} and {\tt
DIGIT} are subsorts of {\tt SIGN} and of {\tt ALPHABET}, and {\tt
SIGN} and {\tt ALPHABET} are direct subsorts of {\tt ANY}.  Hence {\tt
LETTER}, {\tt DIGIT}, and {\tt SIGN} are subsorts of {\tt SIGN} and
{\tt ANY}. {\tt SIGN}, {\tt ALPHABET}, {\tt LETTER}, and {\tt DIGIT}
are subsorts of {\tt ANY}.

\subsection{Variable Symbols}
\label{VariableSymbols} 

Variable symbols are introduced by a $<$variable declaration$>$ in a 
$<$quantification$>$. 

\Ex

{\tt ALL X,Y EX Z:S P(X Y Z)}

The scope of a $<$variable declaration$>$ is the  $<$quantification$>$ following the 
$<$variable declaration$>$ in a $<$quantification$>$.

In its scope each $<$variable symbol$>$ has as rangesort the sort
given by the $<$sort symbol$>$ following the colon in its $<$variable
sort$>$ of the $<$variable declaration$>$. If no $<$variable sort$>$
is present, the rangesort of the $<$variable symbol$>$ is the
predefined sort symbol {\tt ANY}.

\Ex

The expression given in the above example has the following sorts:
rangesort({\tt X}) = rangesort({\tt Y}) = {\tt ANY} and rangesort({\tt
Z}) = {\tt S}.

In each $<$quantification$>$ variable symbols are consistently renamed from 
left to right to resolve conflicts on multiple introductions of variable 
symbols.  

\Ex

{\tt ALL Y,X P(Y)} is the same as {\tt ALL X,Y P(Y)} 

and {\tt ALL X (EX X P(X)) IMPL Q(X)} is the same as {\tt ALL X (EX Y
P(Y)) IMPL Q(X)}.

\subsection{Constant Symbols}
\label{ConstantSymbols} 

Constant symbols are introduced with their first usage 
\begin{itemize}
\item in a $<$type declaration$>$, e.g.\ {\tt TYPE -1,+1:INT}
\item as $<$term$>$, e.g.\ {\tt ALL X P(X A) OR F(C) = D}
\end{itemize}

Each constant symbol has a rangesort: the $<$sort symbol$>$ following the 
colon in the $<$type declaration$>$ which introduces the $<$constant symbol$>$. 

\Ex

For the expressions given above we find
rangesort({\tt -1}) = rangesort({\tt +1}) = {\tt INT}.

The rangesort for a constant symbol which is introduced with its first 
usage as a $<$term$>$ is {\tt ANY}.

Note that in PLL variable symbols are always preceded by a quantifier and 
thereby can always be distinguished from constant symbols. As a 
consequence there is no concept of free variables in PLL. 

\subsection{Function Symbols}
\label{Function Symbols} 

Function symbols are introduced with their first usage in 
\begin{itemize}
\item a $<$type declaration$>$, e.g.\ {\tt TYPE ABS(INT):NAT}
\item an $<$attribute declaration$>$, e.g.\ {\tt ASSOCIATIVE(PLUS)}
\item a $<$term$>$, e.g.\  {\tt ALL X P(F(X)) OR G(X) = A}
\end{itemize}

Each function symbol is associated with a sort symbol for each argument 
position $i$, called its $i$th domainsort, with a natural number, called its 
arity, and with a sort symbol, called its rangesort. 

Function symbols which are introduced by $<$type declaration$>$ have as their 
domainsorts the $<$sort symbol$>$s given on the appropriate positions in the 
list of $<$sort symbol$>$s following the $<$function symbol$>$ in the $<$type 
declaration$>$.

\Ex

For the expression {\tt TYPE PRODUCT(SCALAR VECTOR):VECTOR} we get 
domainsort({\tt PRODUCT}, 1) = {\tt SCALAR} and domainsort({\tt PRODUCT},
2) = {\tt VECTOR}. 

A $<$function symbol$>$ which is introduced by an $<$attribute
declaration$>$ or by its first usage in a $<$term$>$ has {\tt ANY} as
$i$th domainsort for each argument position $i$.

The
rangesort of a $<$function symbol$>$ is defined by the $<$sort
symbol$>$ following the colon in a $<$type declaration$>$. Its
rangesort is {\tt ANY} if the $<$function symbol$>$ is introduced by
an $<$attribute declaration$>$ or by its first usage in a $<$term$>$.

\Ex

For the examples given in the above expressions we get 
rangesort({\tt ABS}) = {\tt NAT},
rangesort({\tt PRODUCT}) = {\tt VECTOR}, and 
rangesort({\tt PLUS}) = rangesort({\tt F}) = rangesort({\tt G}) = {\tt ANY}.

The arity of a function symbol is defined as 
\begin{itemize}
\item the number of sort symbols in the list of $<$sort symbols$>$ 
following the 
$<$function symbol$>$ in the $<$type declaration$>$ which introduces the 
$<$function symbol$>$,
\item as defined in the corresponding attribute declaration for a $<$function symbol$>$ introduced by an $<$attribute 
declaration$>$, or else
\item the number of arguments as its first usage in a $<$term$>$.
\end{itemize}

\Ex

For the expressions given above we get arity({\tt ABS}) = 1,
arity({\tt PLUS}) = 2, and arity({\tt F}) = arity({\tt G}) = 1. 


\subsection{Predicate Symbols} 
\label{PredicateSymbols}
A predicate symbol is introduced with its first usage in 
\begin{itemize}
\item a $<$type declaration$>$, e.g. {\tt TYPE MEMBER(SIGN SET)} 
\item an $<$atom$>$ in a formula, e.g. {\tt EX X,Y P(X Y) AND Q} 
\end{itemize}

Each predicate symbol is associated with a natural number, called its 
arity, and with a sort symbol for each argument position $i$, called its $i$th 
domainsort. 

The arity and domainsort of predicate symbols are determined in the way 
arity and domainsorts are determined for function symbols.

The $<$equality symbol$>$ is a predefined predicate symbol with arity 2 and 
first and second domainsort {\tt ANY}. It is the only predicate symbol which is 
written in infix notation. 

{\tt TRUE} and {\tt FALSE} are predefined predicate symbols with arity
0, which have the obvious meaning.  The phrase unknown symbol in the
subsequent text denotes a string of the terminal alphabet of the
PLL-grammar, which was not used before, that is a new symbol.

\subsection{Semantically Correct Sort Declarations}
\label{SemanticallycorrectSortDeclarations} 

A $<$sort declaration$>$ {\tt SORT S1,...,Sm: T1,...,Tn} is
semantically correct, if 
\begin{itemize}
\item all {\tt Si} and all {\tt Tj} (for all $i=1\ldots m$, $j=1\ldots n$) 
are sort symbols or else are unknown 
symbols and {\tt Si} is a direct subsort of {\tt Tj} or else at least 
one of the symbols {\tt Si} and {\tt Tj} is an unknown symbol.
\end{itemize}


\subsection{Semantically Correct Type Declarations}
\label{SemanticallyCorrectTypeDeclarations} 

A $<$type declaration$>$ T is semantically correct if 
\begin{itemize}
\item T is {\tt TYPE C1,...,Cn:S} and {\tt S} is a sort symbol or else is an unknown 
symbol and for all $i=1\ldots n$ {\tt Ci} is a 
constant symbol with rangesort({\tt Ci}) = {\tt S} or {\tt Ci} is an unknown symbol,
\item T is {\tt TYPE P(S1...Sn)}  and for all $i=1\ldots n$ {\tt Si} is a sort symbol or else is 
an unknown symbol  and {\tt P} is a predicate symbol with arity({\tt
P}) = $n$  and 
domainsort({\tt P}, $i$) = {\tt Si} or else is an unknown symbol, or
\item T is {\tt TYPE F(S1...Sn):S} and for all $i=1\ldots n$ {\tt S} and {\tt Si} are sort symbols or 
else are unknown symbols  and {\tt F} 
is a function symbol with arity({\tt F}) = $n$ , rangesort({\tt F}) = 
{\tt S}  and 
domainsort({\tt Fi}, $i$) = {\tt Si}  or an unknown symbol.
\end{itemize}


\subsection{Semantically Correct Attribute Declarations}
\label{SemanticallyCorrectAttributeDeclarations} 


The $<$attribute declaration$>$s {\tt ASSOCIATIVE(F)} and {\tt AC(F)} are semantically correct if 
\begin{itemize}
\item {\tt F} is a function symbol with arity({\tt F}) = 2, rangesort({\tt F}) = 
domainsort({\tt F}, 1) = domainsort({\tt F}, 2) or else is an unknown 
symbol.
\end{itemize}

The $<$attribute declaration$>$ {\tt AC1(F F1)} is semantically correct if 
\begin{itemize}
\item {\tt F} is a function symbol with arity({\tt F}) = 2, rangesort({\tt F}) = 
domainsort({\tt F}, 1) = domainsort({\tt F}, 2) 
and \item {\tt F1} is  a constant symbol with sort equal to the rangesort
of {\tt F} or both are unknown symbols.
\end{itemize}

The $<$attribute declaration$>$ {\tt AG(F F- F1)} is semantically correct if 
\begin{itemize}
\item {\tt F} is a function symbol with arity({\tt F}) = 2,
rangesort({\tt F}) = domainsort({\tt F}, 1) = domainsort({\tt F}, 2),
\item {\tt F-} is a function symbol with arity({\tt F-}) = 1,
rangesort({\tt F}) = domainsort({\tt F-}, 1) = rangesort({\tt F-}), and
\item {\tt F1} is  a constant symbol with sort equal to the rangesort
of {\tt F} or all three are unknown symbols.
\end{itemize}



The $<$attribute declarations$>$s {\tt REFLEXIVE(P)}, {\tt
IRREFLEXIVE(P)}, and {\tt SYMMETRIC(P)} are semantically correct if

\begin{itemize}
\item {\tt P} is a predicate symbol with arity({\tt P}) = 2 and 
domainsort({\tt P}, 1) = domainsort({\tt P}, 2)  or else {\tt P} is an unknown symbol.
\end{itemize}



\subsection{Semantically Correct Terms, Atoms, and Quantifications}
\label{SemanticallyCorrectTermsAtomsandQuantifications}


The sort of a term T, denoted sort(T), is the rangesort of T, if T is a 
variable or constant symbol, and else is the rangesort of the outermost 
function symbol of T. 
A $<$term$>$ T is semantically correct if 
\begin{itemize}
\item T is a constant symbol, a variable symbol, or an unknown symbol
or
\item T is {\tt F(T1...Tn)} and for all $i=1\ldots n$, {\tt Ti} is a semantically correct term,
{\tt F} is a 	function symbol with arity({\tt F}) = $n$  and 
sort({\tt Ti}) is a subsort of domainsort({\tt F}, $i$)  or else {\tt F} is an 
unknown symbol.
\end{itemize}


An $<$atom$>$ A is semantically correct if 
\begin{itemize}
\item A is a predicate symbol with arity(A)=0  or A is an unknown 
symbol,
\item A is {\tt P(T1...Tn)} and for all $i=1\ldots n$, {\tt Ti} is a semantically correct term, {\tt P}
is a predicate symbol with arity({\tt P}) = $n$  and 
sort({\tt Ti}) is a subsort of domainsort({\tt P}, $i$)  or else {\tt P} is an 
unknown symbol , or
\item A is {\tt T1 = T2} and {\tt T1} and {\tt T2} are semantically 
correct terms and {\tt =} is an $<$equality symbol$>$. 
\end{itemize}



A $<$quantification$>$ Q is semantically correct if
\begin{itemize}
\item Q is {\tt ALL X I} or {\tt EX X I} and {\tt X} is a variable symbol or an unknown 
symbol and each atom in Q is semantically correct. 
\end{itemize}


\section{Errors Detected by the Compiler}
\label{ErrorsdetectedbytheCompiler}

The PLL compiler of the ATP system checks each input for syntactical
and semantical correctness. An input containing signs which are not
member of the terminal alphabet is responded by a message {\tt Symbol}
{\tt error:} {\tt $<$xxx$>$} {\tt is} {\tt no} {\tt admissible} {\tt
symbol}, where {\tt $<$xxx$>$} is a sign which is not member of the terminal
alphabet.

For a syntactically incorrect input, the compiler will respond {\tt
Syntax} {\tt error:} {\tt $<$xxx$>$} {\tt not} {\tt accepted,} {\tt
unexamined} {\tt remainder} {\tt of} {\tt the} {\tt input:} {\tt
$<$zzz$>$}, where $<${\tt xxx}$>$ is the sign which causes the
syntactical incorrectness and {\tt $<$zzz$>$} is the not analysed
remainder of the given input.

For a syntactically correct but semantically incorrect input, the
compiler responds {\tt Semantic error:} {\tt $<$message$>$,} {\tt
unexamined} {\tt remainder} {\tt of} {\tt the} {\tt input:} {\tt
$<$zzz$>$}, where {\tt $<$message$>$} is an error message explaining
the kind of the semantic error and {\tt $<$zzz$>$} is the not analysed
remainder of the given input.


\section{Particularities of the Input Routines} 
\label{ParticularitiesoftheInputRoutines}

Since the whole ATP system is a Lisp program, the special features 
of the Lisp input routines have to be taken into account, i.e.

\begin{itemize}
\item {\tt ()} is read as {\tt NIL}
\item {\tt 'X} is read as {\tt (QUOTE X)}
\item each input has to contain an even number of the string indicators {\tt "}
\item a sequence of blanks is read as one blank (except in a string) 
\item {\tt +} is read as a blank if it is followed by a sequence of
digits, e.g. {\tt +4711} is read as {\tt 4711}
\item a sequence of zeroes is read as a zero, unless the sequence is preceded 
by non-zero sign, e.g. {\tt 007} is read as {\tt 7}
\end{itemize}


In Lisp each of the following characters separates expressions: 

\begin{itemize}
\item a blank
\item a  paranthesis,  {\tt )} or {\tt (}
\item the quote sign, {\tt '}
\item the string indicator, {\tt "}
\end{itemize}

Signs acting as separators in PLL are 

\begin{itemize}
\item all Lisp separators 
\item the colon, i.e.\ {\tt  X:Y} is the same as {\tt X : Y}
\item the comma, i.e.\ {\tt X,Y} is the same as {\tt X , Y}
\end{itemize}

No other signs than those mentioned here are separating, that is,
they can not be detected inside a symbol. {\tt X=Y}, for example, denotes
the symbol {\tt X=Y} and not the equational atom {\tt X = Y}.
